\section{Spanners in the Probability Density Setting: Full
  Proof}\label{ap:distribution-spanner}

We prove Theorem~\ref{thm:distribution-spanner} in full.
Through this section, we assume that $D$ is a probability density
function with support on smooth connected compact manifold with intrinsic dimension
$d$ embedded in ambient space $\mathbb{R}^s$, 
with smooth boundary of bounded curvature. This
probability density function is further assumed to be bounded
above and below, and to be Lipschitz. For simplicity, we assume that
$s=d$, and we can prove all our results when $s > d$ by taking
coordinate charts from the manifold into Euclidean space.  We
will show at the end of the section that if the distribution is
supported on a convex set of full dimension in the ambient space,
then the $k$-NN graph is Gabriel for the same $k$. It is not
difficult to see that Gabriel graphs are $1$-spanners of the
edge-squared metric~\cite{SridharMaster}.

\begin{lemma}
  Let $\M$ be a compact object in $\mathbb{R}^d$, whose
  boundary is a smooth manifold of dimension $d-1$ with bounded
  curvature.
  Let $\B$ be any ball with sufficiently small radius
  $r_B$ with center in $\M$, that intersects the boundary of $D$
  at some point $x$.
  Let $H$ be the halfspace tangent to $\M$ at $x$ containing the center
  of the ball.

For any point $Q \in \M$, let $Q'$ be the point in $B$
closest to $Q$. If $d(Q', H) / r_B > c$ for arbitrary constant $c$,
then $d(Q, H) \geq c'$ for some constant $c'$.
\end{lemma}

This is a basic fact about the smoothness and bounded curvature
of the boundary.

\begin{lemma} Pick $n$ points from $D$. W.h.p, any two points in
  $Support(D)$ with Euclidean distance $ \geq \Omega(1)$ 
  have nearest neighbor metric of $o(1)$.
\end{lemma}

This is implicit in~\cite{hwang2016}.
\begin{lemma} For any ball $\B$ with center $O$ and any point $Q'$ on the
  boundary of $B$, let $B_{Q'O}$ be the ball with diameter $Q'O$.
  Let $H$ be any halfspace containing $O$.
  If $d(Q', H) / r_B \leq c$ for some constant
  $c$ possibly depending on the dimension $d$, then
  $\vol(\B_{Q'O} \cap H) \geq \frac{1-c'}{2^d} \vol(\B \cap H)$
  for some
  constant $c'$, where $c'$ goes to $0$ as $c$ goes to $0$.
\end{lemma}

\begin{proof} First, let us consider the case where $d(Q' H) =
0$, that is, $Q'$ is contained in halfspace $H'$.  In this case,
  dilating $B_{Q'O} \cap H$ by a factor of $2$ about point $Q'$
  gives a superset of $B\cap H$, as $B_{Q'O}$ maps to $B$ and $H$
  maps to a halfspace strictly containing $H$. In this case,
  $\vol(\B_{Q'O} \cap H) \geq \frac{1}{2^d} \vol(\B \cap H)$ as
  desired.
  The case when
  $d(Q', H)/r_B$ is bounded follows in a straightforward manner.
  \end{proof}


This leads us to our following theorem:
\begin{theorem} For any $n$ point set $P$ picked i.i.d from $D$, consider any
  point $O$. Let $\B$ be the $k$-NN ball of $O$.  
  Let $Q \in Support(D)$ be any point outside 
  $\B$, and let the closest point to $Q$ in $\B$ be $Q'$. For a
  point $x$ inside $B$ on the boundary of $D$ (assuming such a
  point exists), let $H$ be the tangent halfplane containing the
  center of $\B$.

  Then: either $d(Q', H) /
  r_B \leq c'$ for some constant $c'$ or there exists a constant
  $c$ where $|QO| > c$. Here, $c$ and $c'$ are
  independent of the number of points chosen, and $c'$ can be set
  arbitrarily small.

  In the latter case, w.h.p. $QO$ is not in the edge-squared
  $1$-spanner. In the former case, setting $c'$ to be a very
  small constant $\epsilon$ lets us say:

  \begin{align} \label{eq:vol}
    \vol(\B_{Q'O} \cap H) \geq \frac{1-\epsilon}{2^d} \vol(\B
    \cap H),
  \end{align}
  or equivalently:
  \begin{align} \label{eq:prob}
    & \prob{x\sim D}{x \in \B_{QO} | x \in \B}
  \\
    \label{eq:prob-2}
    \geq & \prob{x\sim D}{x \in \B_{Q'O} | x \in \B}
  \\ \label{eq:prob-3}
    \geq & \frac{1-\eps-o(1)}{2^d}
  \end{align}
\end{theorem}
Expression~\ref{eq:prob-2} $>$ Expression~\ref{eq:prob-3} follows from Equation~\ref{eq:vol}, and
the fact that the radius of the $k$-NN ball goes to $0$ as $n$
gets large, and thus the probability density of sampling $x$ from
$D$ conditioned on $x$ being in $\B$ approaches the uniform 
density in $\B \cap Support(D)$. Also, $B \cap H$ approaches $B \cap Support(D)$ as the
radius of $B$ goes to $0$. 

Expression~\ref{eq:prob} $>$
Expression~\ref{eq:prob-2} since $\B_{QO} \supset B_{Q'O}$.
(Here, the $k$-NN ball $B$ w.r.t. point $O$ is defined as the ball
centered at $O$ with radius equal to the distance of the $k^{th}$
nearest neighbor to $O$).

Note that the $k-1$ nearest neighbors of $O$, conditioned only on
the radius of $B$, are distributed
equivalently to $k-1$ i.i.d samples of $D$ conditioned on containment
in $\B$. It follows
that for any point $Q$ outside $B$ and in the support of $D$,
where $|QO|< c$:
  
\begin{align}
  \prob{P \sim D^k}{QO \text{ is not Gabriel w.r.t. $P$} | Q \not\in B}
  \geq 1- \left(1 - \frac{1-\eps-o(1) }{2^d} \right)^k
\end{align}

Thus, setting $\epsilon = 0.1$ and $ k > O(\log n/2^d) $, and
factoring in the case where $|QO| > c$, then w.h.p.:
\[
  \prob{P \sim D^k}{QO \text{ is not critical w.r.t. $P$} | Q \not\in B}
  \] 
Here, we recall that an edge $AB$ is Gabriel with respect to a
point set $P$ if and only if $\B_{AB}$ does not contain any points
in $P$. Note that every non-Gabriel edge is non-critical, where a
critical edge is an edge that must be in the $1$-spanner (as
in Definition~\ref{def:critical}).
 Thus taking the union bound over $Q, O \in P$ gives us that
no edge outside the $k$-NN graph is critical w.h.p, and thus the
$k$-NN graph contains all critical edges and is a $1$-spanner
w.h.p.

This proves Theorem~\ref{thm:distribution-spanner} when the
support of $D$ has the same intrinsic dimension as the ambient
space. If the support of $D$ has  dimension $d < d'$ (where $d'$
is the ambient dimension of the space), simply take coordinate charts from
$D$ onto $\mathbb{R}^d$ and the previous arguments will still
carry through
.
We should note that if no point $x$ inside $B$ on the boundary of
$D$ exists, then we can ignore $H$ and all the steps of
the proof still follow.
