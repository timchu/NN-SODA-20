\subsection{Construction of $\Conv$}

In this section, we prove we Lemma~\ref{lem:flow}. We construct a function
$\Conv:\mathbb{R}^n \rightarrow \mathbb{R}^{2^n}$ assigning every point in
Euclidean space to a vector, representing a convex combination of vertices
in $G'$. We will build $\Conv$ separately for each Voronoi cell, and show
it is piecewise continuous across boundaries of Voronoi cells. 
However, to simplify our arguments, we further divide up each Voronoi cell
into simplices similar to the dissection in a barycentric subdivision.
However, rather than barycenters, we use circumcenters, so we are
more-precisely dividing up the $q$-screw simplex with a circumcentric
subdivision. \tim{define circumcentric/barycentric subdivision}

Let $p_{S}$ be the circumcenter of points $\{p_s | s \in S\}$ when $S
\subset \{1, 2, \ldots n\}$. A simplicial cell in the circumcentric subdivision is
defined by a permutation $a_0, a_1, \ldots a_{n-1}$ of $\{0, 1, \ldots
n-1\}$ as follows: let
$A_i = \{a_0, a_1, \ldots a_{i}\}$. Then the vertices $\{p_{A_i} | 0 \leq i
< n\}$ are the vertices of the simplicial cell.  \tim{Move this definition
upwards somewhere?} Each Voronoi cell is the disjoint~\tim{not exactly
disjoint} union of some of
these cells, and the cells partition the simplex \tim{This isn't quite
true: its only true for special geometry cells, but we don't particularly
care... how can I make this point clear?}. Our general strategy is to
create $\Conv$ on each simplicial cell, and show that is piecewise
continuous across the boundaries of simplicial cells.

To simplify our notation, we define $\pr_i$ to be $p_{A_i}$. \tim{Make sure
all points are $p$, not $x$. $\xr_k$ is later used to coordinate-wise split
up $x$}.

%Define $\pr, \xr, \Rr, \Convr$ here. Points $\pr$ numbered from $0$ to $k-1$.
%Here, $\xr$ is numbered $1$ through $k-1$. $B$ is $0$ through $k-1$
%indexed.

\tim{Cut this next paragraph?}
Our general strategy is to create $\Conv$ piecewise, where we define it on
each Voronoi subdivision bounded by the vertices $ p_{a_0a_1\ldots a_t}$
for all $0 \leq t \leq k$. By the construction of these points,
$p_{a_0}p_{a_0a_1}$ is perpendicular to $p_{a_0a_1}p_{a_0a_1a_2}$, and so
forth.
To simplify our notation, we will denote $p_{a_0a_1\ldots a_t}$ as $\pr_t$,
for any $0 \leq t \leq k$. 

By the construction of circumcentric subdivisions, 
the line $\pr_{i}\pr_{i+1}$ is perpendicular to
$\pr_{i+1}\pr_{i+2}$ for all $i$. These lines define a natural
orthonormal coordinate axis. Thus,
for any $\xr$ in the convex hull of $\pr_i$, we can write $\xr$ in
coordinates $(\xr_0, \xr_1, \ldots \xr_k)$, where the $i^{th}$ coordinate
axis is parallel to $\pr_i \pr_{i+1}$.

Next, we introduce how $\Convr$ is defined on the vertices $\pr_{i}$. Here,
$\Convr_i$ is shorthand for the value of $\Convr$ on $\pr_{i}$.

\begin{align}
\Convr(\xr)_i = \frac{
\left(\sum_{s=1}^i {\xr_s}^2\right)^{q/2} -\left(\sum_{s=1}^{i-1}
{\xr_s}^2\right)^{q/2}
}
{
\Rr_i^q - \Rr_{i-1}^q
} - \sum_{i < j < n} \Convr(\xr)_j
\end{align}

for all $1 \leq i \leq n-1$, and \tim{make sure you replace all k's with
n's, and claim that you're dealing with simplices of full dimension. You
may want to say that the proof is counterintuitive since simplices of full
dimension are usually considered harder?}

\[ \Convr(\xr)_0 = 1 - \sum_{1 \leq j < n} \Convr(\xr)_j
\]

The key feature about $\Convr$ is that

\[\sum_{j=i}^k \Convr(\xr)_j= \frac{
\left(\sum_{s=1}^i {\xr_s}^2\right)^{q/2} -\left(\sum_{s=1}^{i-1}
{\xr_s}^2\right)^{q/2}
}
{
\Rr_i^q - \Rr_{i-1}^q
}\]

for all $i > 0$, and for $i = 0$ the LHS evaluates to $1$.

Since we defined this function piecewise, we need to check that this
function is piecewise continuous.

\begin{lemma} If $\xr$ is on a face of $\pr_0, \ldots \pr_n$, then
$\Convr(\xr)$ has non-zero coordinates only on that face. Furthermore, the
coordinates depend only on SOMETHING.

\end{lemma}

\begin{proof}

\end{proof}

Now we are ready to prove our core lemma:

\begin{lemma} For $x$ and $x+\Delta(x)$ in the convex hull of $\pr_1,
\ldots \pr_k$, the distance:

\[ \|x-\pr_1\|^{q-1} \cdot \|\Delta(x)\| \geq
F\left(\Convr(x+\Delta(x))-\Convr(x)\right)\]

where $F(d)$ is the unique cost of a flow satisfying demand $d \in
\mathbb{R}^k$ on $\Gr$.

\end{lemma}

Here, the left hand side represents the $q$-NN cost of a path piece from
$x$ to $x+\Delta(x)$, and the right hand side is the unique cost of the
induced flow on graph $G'$, with the restriction that the flow is only
nonzero on the vertices $\pr_i\pr_{i+1}$ for any $0 \leq i < n$. This flow
is unique since we forced our flow to be non-zero only on the edges
$\pr_i\pr_{i+1}$, which form a line graph; and for any set of demands on
vertices of a line, there is a unique flow satisfying those demands.

\begin{proof} For any edge $\pr_i\pr_{i+1}$, the cost of a flow (satisfying
some set of demands whose sum is $0$) on that edge is the absolute value of
the sum of the demands on vertices $\pr_{i+1} \pr_{i+2}, \ldots \pr_k$,
multiplied by the cost of the edge from $\pr_i$ to $\pr_{i+1}$. This
quantity comes out to be:

\begin{align} &\left(\Rr_{i+1}^q-\Rr_{i}^q \right) \sum_{j=i+1} \Convr(\xr)_j
\\
\label{eq:flow-cost}
&=\left(\sum_{s=1}^i {\xr_s}^2\right)^{q/2} -\left(\sum_{s=1}^{i-1}
{\xr_s}^2\right)^{q/2}.
\end{align}

As $\Delta(x)$ goes to $0$, the change in Expression~\ref{eq:flow-cost} is

\begin{align}
\begin{split}
\label{eq:flow-cost-on-edge}
&
\left(q\xr_0 \left(\sum_{s=1}^i \xr_s^2 \right)^{q/2-1} -
q\xr_0\left(\sum_{s=1}^{i-1} \xr_s^2 \right)^{q/2-1} \right) \Delta(\xr)_0
\\
&+
\left(q\xr_1 \left(\sum_{s=1}^i \xr_s^2 \right)^{q/2-1} -
q\xr_1\left(\sum_{s=1}^{i-1}\xr_s^2 \right)^{q/2-1} \right) \Delta(\xr)_1
\\
&+ \ldots
\\
&+
\left(q\xr_{i-1} \left(\sum_{s=1}^i \xr_s^2 \right)^{q/2-1} -
q\xr_1\left(\sum_{s=1}^{i-1}\xr_s^2 \right)^{q/2-1}
\right)\Delta(\xr)_{i-1}
\\
&+
\left(q\xr_i \left(\sum_{s=1}^i \xr_s^2
\right)^{q/2-1}\right)\Delta(\xr)_i,
\end{split}
\end{align}
Since 
\[
q\xr_j\left(\sum_{s=0}^i \xr_s^2\right)^{q/2-1} - q\xr_j\left(\sum_{s=0}^{i-1}
\xr_s^2 \right)^{q/2-1}\]
is always non-negative (only when $q \leq 2$), we get that the absolute value of
Expression~\ref{eq:flow-cost-on-edge} is bounded above by:
\begin{align}
\begin{split}
\label{eq:flow-cost-upper-bound}
&
\left(q\xr_0 \left(\sum_{s=1}^i \xr_s^2 \right)^{q/2-1} -
q\xr_0\left(\sum_{s=1}^{i-1} \xr_s^2 \right)^{q/2-1} \right)
|\Delta(\xr)_0|
\\
&+
\left(q\xr_1 \left(\sum_{s=1}^i \xr_s^2 \right)^{q/2-1} -
q\xr_1\left(\sum_{s=1}^{i-1}\xr_s^2 \right)^{q/2-1} \right) |\Delta(\xr)_1|
\\
&+ \ldots
\\
&+
\left(q\xr_{i-1} \left(\sum_{s=1}^i \xr_s^2 \right)^{q/2-1} -
q\xr_1\left(\sum_{s=1}^{i-1}\xr_s^2 \right)^{q/2-1}
\right)|\Delta(\xr)_{i-1}|
\\
&+
\left(q\xr_i \left(\sum_{s=1}^i \xr_s^2
\right)^{q/2-1}\right)|\Delta(\xr)_i|,
\end{split}
\end{align}

Expression~\ref{eq:flow-cost-upper-bound} is an upper bound on the cost of
a flow along edge $\vr_i\vr_{i+1}$\tim{Is this the right indexing for
edges?} induced by a path from $x$ to $x+\Delta(x)$. Now we sum this across all $i$ to get an overall cost upper bound, and group by
$\Delta(\xr)_i$ for fixed i. The sum telescopes beautifully, and we get:
\begin{align}
&\left(q\xr_0\left(\sum_{s=1}^k \xr_s^2\right)^{q/2-1}\right)| \Delta(\xr)_0|
\\&+
|q\xr_1\left(\sum_{s=1}^k \left(\xr_s^2\right)^{q/2-1}\right)| \Delta(\xr)_1|
\ldots
\\& +
|q\xr_k\left(\sum_{s=1}^k (\xr_s^2)^{q/2-1}\right)| \Delta(\xr)_k|
\end{align}
This expression, by Cauchy Schwarz, is upper bounded by
\[
\sqrt{\sum_{s=0}^k \Delta(\xr)_s^2} \cdot \left(q\sqrt{\sum_{s=0}^k
\xr_s^{q-1}}\right)
\]
Which is exactly the $q$-NN distance.

\end{proof}

Note that this function is piecewise continuous on the boundary. Therefore,
we have shown that the $q$-NN cost of any path piece is less than the
min-cost flow on $G’$ satisfying
$\Conv\left(x+\Delta(x)\right)-\Conv\left(x\right)$ for infinitesimal
$\Delta(x)$, as desired.

So far, the only property our flow construction used is that the points
$x_0, x_1, \ldots x_n$ have Voronoi subdivisions defined by $\pr_{a_0}$,
$\pr_{a_0a_1}$, \ldots $\pr_{a_0a_1\ldots a_k}$, for some $a_0, ldots a_k
\subset \{0, 1, \ldots n\}$. (DOES THIS WORK FOR ANY GEOMETRY, OR DO I NEED
THE INTERIOR CIRCUMCENTER PROPERTY?).

Thus, we have proven a core lemma:

\begin{lemma}\label{lem:qNN-GPrime}

The $q$-NN distance between two points in a point set is lower bounded by
the shortest path between the two corresponding points in $G’$. Here, $G’$
is constructed as in Definition~\ref{def:GPrime}

\end{lemma}

We now prove the following two lemmas, completing our proof of
Theorem~\ref{thm:NN} and Theorem~\ref{thm:qNN} respectively.

\begin{lemma}\label{lem:edge-squared-GPrime} Let $G$ be the edge-squared graph
(DEFINE), and let $G’$ be defined as in Definition~\ref{def:GPrime} for $q=2$.
The shortest path in $G’$ is the same as the shortest path in $G$, when the
initial point set generating $G$ and $G’$ is a $2$-screw simplex.

\end{lemma}

\begin{lemma}\label{lem:q-edge-power-GPrime} Let $q > 2$. Let $G$ be the
$q$-edge power graph, and $G’$ be defined as in Definition~\ref{def:GPrime}
(MAKE SURE THE DEFINITION IS Q DEPENDENT). The shortest path in $G’$ is the
same as the shortest path in $G$, when the initial point set generating $G$
and $G’$ is a $q$-screw simplex with $4$ points.

\end{lemma}

Combined with Theorem~\ref{thm:screw-simplex-reduction},
Lemmas~\ref{edge-squared-GPrime} and~\ref{q-edge-power-GPrime} prove
Theorems~\ref{thm:NN} and~\ref{thm:qNN} respectively. Moreover, we make the
following conjecture, which we have some computational evidence for (See
Appendix~\ref{} for details):

\begin{conjecture}\label{conj:qNN}

For $q>2$, let $G$ and $G’$ be defined as in
Lemma~\ref{lem:q-edge-power-GPrime}. Then the shortest path in $G’$ is the same
as the shortest path in $G$.

\end{conjecture}

If this were true, it would prove that the $q$-edge power metric and the
$q$-NN metric were equal for all $q>2$.
