\subsection{Persistent Homology of the Nearest-neighbor Geodesic Distance}
  \label{sec:persistence}

% \tim{This section should be prefaced somewhere -- here or in a previous
% section -- with a statement as to why the Nearest Neighbor metric may
% still be of independent interest. My main claim is that it's useful
% since it's defined on all points rather than just two.}
% 
% \tim{My main issue with this section is that I don't have a clean
% theorem saying how to compute persistent homology, both in ambient and
% intrinsic setting. Having one or two top-level theorems that say this
% would be great, rather than having it be written in the exposition. As
% it stands, I have no clue how Lemma 4.5 is used, or Lemma 4.6, to
% compute the homologies.}
  In this section, we show how to compute the so-called persistent homology~\cite{edelsbrunner02topological} of the Nearest Neighbor Metric distance in two different ways, one ambient and the other intrinsic.
  The latter relies on Theorem~\ref{thm:NN} and would be quite surprising without it.

  Persistent homology is a popular tool in computational geometry and topology to ascribe quantitative topological invariants to spaces that are stable with respect to perturbation of the input.
  In particular, it's possible to compare the so-called persistence diagram of a function defined on a sample to that of the complete space~\cite{chazal08towards}.
  These two aspects of persistence theory---the intrinsic nature of topological invariants and the ability to rigorously compare the discrete and the continuous---are both also present in our theory of Nearest Neighbor Metric distances.
  Indeed, the primary motivation for studying these metrics was to use them as inputs to persistence computations for problems such as persistence-based clustering~\cite{chazal13persistence} or metric graph reconstruction~\cite{aanjaneya12metric}.

  The input for persistence computation is a \emph{filtration}---a nested sequence of spaces, usually parameterized by a real number $\alpha\ge 0$.
  The output is a set of points in the plane called a \emph{persistence diagram} that encodes the birth and death of topological features like connected components, holes, and voids.

  \paragraph*{The Ambient Persistent Homology}

    Perhaps the most popular filtration to consider on a Euclidean space is the sublevel set filtration of the distance to a sample $P$.
    This filtration is $(F_\alpha)_{\alpha\ge 0}$, where
    \[
      F_\alpha := \{x\in \R^d \mid \distto_P(x) \le \alpha\},
    \]
    for all $\alpha \ge 0$.
    If one wanted to consider instead the Nearest Neighbor Metric distance $\dist_N$, one gets instead a filtration $(G_\alpha)_{\alpha\ge 0}$, where
    \[
      G_\alpha := \{x\in \R^d \mid \min_{p_\in P} \dist_N(x,p) \le \alpha\},
    \]
    for all $\alpha \ge 0$.

    Both the filtrations $(F_\alpha)$ and $(G_\alpha)$ are unions of metric balls.
    In the former, they are Euclidean.
    In the latter, they are the metric balls of $\dist_N$.
    These balls can look very different, for example, for $\dist_N$, the metric balls are likely not even convex.
    However, these filtrations are very closely related.
    \begin{lemma}
      For all $\alpha \ge 0$, $F_\alpha = G_{2\alpha^2}$.
    \end{lemma}
    \begin{proof}
      The key to this exercise is to observe that the nearest point $p \in P$ to a point $x$ is also the point that minimizes $\dist_N(x,p)$.
      To prove this, we will show that for any $p\in P$ and any path $\gamma\in \ourpath(x,p)$, we have $\len(\gamma)\ge \frac{1}{2}\distto_P(x)^2$.
      Consider any such $x$, $p$, and $\gamma$.
      The euclidean length of $\gamma$ must be at least $\distto_P(x)$, so we will assume that $\|\gamma'\| = \distto_P(x)$ and will prove the lower bound on the subpath starting at $x$ of length exactly $\distto_P(x)$.
      This will imply a lower bound on the whole path.
      Because $\distto_P$ is $1$-Lipschitz, we have $\distto_P(\gamma(t))\ge (1-t)\distto_P(x)$ for all $t\in [0,1]$.  It follows that
      \[
        \len(\gamma) = \int_0^1 \distto_P(\gamma(t))\|\gamma'(t)\|dt
          \ge \distto_P(x)^2 \int_0^1 (1-t)dt
          =  \frac{1}{2}\distto_P(x)^2
      \]
      The bound above applies to any path from $x$ to a point $p\in P$, and so,
      \[
        \dist_N(x,p) = 4 \inf_{\gamma\in \ourpath(x,p) \len(\gamma)} \ge 2\distto_P(x).
      \]
      If $p$ is the nearest neighbor of $x$ in $P$, then $\dist_N(x,p) = 2\distto_P(x)$, by taking the path to be a straight line.
      It follows that $\min_{p\in P}\dist_N(x,p) = 2\distto_P(x)$.
    \end{proof}

    The preceding lemma shows that the two filtrations are equal up to a monotone change in parameters.
    By standard results in persistent homology, this means that their persistence diagrams are also equal up to the same change in parameters.
    This means that one could use standard techniques such as $\alpha$-complexes~\cite{edelsbrunner02topological} to compute the persistence diagram of the Euclidean distance and convert it to the Nearest Neighbor Metric distance afterwards.
    Moreover, one observes that the same equivalence will hold for variants of the Nearest Neighbor Metric distance that take other powers of the distance.

    % (some are considered in Section~\ref{sec:higher_powers}).


  \paragraph*{Intrinsic Persistent Homology}
    Recently, several researchers have considered intrinsic nerve complexes on metric data, especially data coming from metric graphs~\cite{adamszek16nerve,gasparovic17complete}.
    These complexes are defined in terms of the intersections of metric balls in the input.
    The vertex set is the input point set.
    The edges at scale $\alpha$ are pairs of points whose $\alpha$-radius balls intersect.
    In the intrinsic \v Cech complex, triangles are defined for three way intersections, tetrahedra for four-way intersections, etc.

    In Euclidean settings, little attention was given to the difference between the intrinsic and the ambient persistence, because a classic result, the Nerve Theorem~\cite{borsuk48imbedding}, and its persistent version~\cite{chazal08towards} guaranteed there is no difference.
    The Nerve theorem, however, requires the common intersections to be contractible, a property easily satisfied by convex sets such as Euclidean balls.
    However, in many other topological metric spaces, the metric
    balls might not be so well-behaved.
    In particular, the Nearest Neighbor Metric distance has metric balls which may take on very strange shapes, depending on the density of the sample.
    This is similarly true for graph metrics.
    So, in these cases, there is a difference between the information in the ambient and the intrinsic persistent homology.

    \begin{theorem}
      Let $P\subset \R^d$ be finite and let $\dist_N$ be the Nearest Neighbor Metric distance with respect to $P$.
      The edges of the intrinsic \v Cech filtration with respect to $\dist_N$ can be computed exactly in polynomial time.
    \end{theorem}
    \begin{proof}
      The statement is equivalent to the claim that $\dist_N$ can be computed exactly between pairs of points of $P$, a corollary of Theorem~\ref{thm:equality}.
      Two radius $\alpha$ balls will intersect if and only of the distance between their centers is at most $2\alpha$.
      The bound on the distance necessarily implies a path and the common intersection will be the midpoint of the path.
    \end{proof}
