  \section{Nearest Neighbor Metric and Edge-Power Metrics relate to Single Linkage Clustering, Level Sets, and k-Centers
  clustering}\label{ap:clustering-link}

  Many popular clustering algorithms, including $k$-centers, $k$-means,
  and $k$-medians clustering, use Euclidean distance as a measure of
  distance between points in $\mathbb{R}^d$. These methods are useful
  when clusters are spherical and well-separated. However, it is believed by
  practitioners that
  density-sensitive distances more accurately capture intrinsic
  distances between data \cite{alamgir12shortest}.

  The celebrated single-linkage clustering algorithm~\cite{Gower1969, Yaro2017},
  which is clustering based on an MST, is a widely used tool in
  machine learning, and gets around many of the problems of the
 Euclidean distance clustering. In single-linkage clustering,
   two points are considered similar if the maximum length
   edge on the path between them in the MST is small. This turns out to be equivalent to 
  computing the normalized $\infty$-power metric between the two
  points. Therefore, single linkage clustering can be seen
  as clustering using the normalized $\infty$-power metric.
  Generally, normalized $p$-power metrics can be seen as an
  intermediary between Euclidean distances ($1$-power metrics) and
  Euclidean MST-based clustering.

  Clustering with $p$-power metric relates to another popular
 clustering method in machine learning, known as level-set clustering.
  Loosely speaking, level set clustering involves finding an estimate
  for the probability density that points are drawn from, finding a
  cut threshold $t$, and then taking as clusters all regions with
  probability density $ > t$. Level set clustering
  has appeared in many incarnations~\cite{Wishart69, Stuetzle2003, Stuetzle2007}, including the celebrated and
  widely used DBScan method~\cite{Ester1996} and its
  considerable number of variations~\cite{OPTICS96}.
It is known
  that level-set clustering is related to single-linkage clustering, as
  the latter is an approximation of the former~\cite{Wishart69,
  Stuetzle2007}.  
  Level-set methods
  have the advantage that they can find arbitrarily shaped
  clusters~\cite{Ester1996}, but can cause two points that are very
  close in Euclidean distance to be considered far apart. 

  Clustering with the $p$-power metric incorporates the
  advantages of both Euclidean distance clustering and level set
  clustering, as it is both density-sensitive and takes into account
  overall
  Euclidean distance between two points.
  Here, $p$ can be toggled to change the sensitivity of the metric
  to the underlying density.
  As the number of samples
  drawn from our probability density grows large, it has been proven that the behavior of
  normalized $p$-power metrics converges to a natural geodesic distance on the
  underlying probability density~\cite{hwang2016}. Clustering with
  this geodesic distance for $p=1$ is exactly Euclidean clustering, and for
  $p=\infty$ is exactly the level set method. 
   Thus, clustering with $p$-power metric converges to
  a clustering method that smoothly interpolates between
  Euclidean-distance clustering and level set clustering. 
% \tim{Maybe you want to mention the self-sparsifying graph when you raise
% $p$ from $1$ to $\infty$}
