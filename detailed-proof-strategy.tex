\subsection{Detailed Proof Strategy}

Given points $x_0, x_1, \ldots x_n$, our strategy is to build a graph $G'$ with
vertices $v_0, v_1, \ldots v_{n-1}$, and $v_S$ for any subset $S$ of
$\{0,1,2,3,4 \ldots n-1 \}$. Here, $v_S$ is a vertex representing the
circumcenter of the vertex set $\{x_s: s \in S\}$. Graph $G'$ then has some
edges (connected in a fashion to be detailed later), and costs on these edges.
Graph $G'$ is similar to the barycentric subdivision on a graph, used
in~\cite{}.

Tto lower bound any nearest neighbor path in our original point set, we will
lower bound it by some unit flow (DFEINE) from $v_0$ to $v_n$ in $G'$. However,
note that the unit flow from $v_0$ to $v_n$ is lower bounded by the shortest
path (where lengths of edges are the same as their costs) from $v_0$ to $v_n$
in $G'$. If the shortest path in $G'$ is equal to the $q$-edge power cost of
going from $x_0$ to $x_n$, then we have completed our proof.

To lower bound the cost of the $q$-NN metric with a unit flow, we aim to build
a potential function $\Conv:\mathbb{R}^n \rightarrow \mathbb{R}^{2^n}$ such
that $\Conv(p)$ is a vector whose entries sum to one. Here, $\Conv(p)$
represents a convex combination of the vertices of $G’$. We construct $\Conv$
such that $\Conv(v_S) = e_S$, where $e_S$ represents the unit vector with $1$
in the coordinate indexed by $S$, and $0$ elsewhere. Additionally, $\Conv$ has
the property that the $q$-NN cost of any path from $x \in \mathbb{R}^n$ to
$x+\Delta(x) \in \mathbb{R}^n$ (with respect to point set $x_0, \ldots x_n$) is
bounded below by the min cost flow on $G’$ satisfying demands
$\Conv(x+\Delta(x)) - \Conv(x)$. If we can construct such a function $\Conv$,
then the $q$-NN path from $v_0$ to $v_n$ is thus lower bounded by the min cost
unit flow from $v_0$ to $v_n$. Therefore, we have successfully lower bounded
the $q$-NN cost of a path from $x_0$ to $x_n$ with the shortest path in $G’$.

Therefore, the remainder of the section is devoted to two tasks: first,
constructing a $G’$ such that the shortest path in $G’$ is equal to the
$q$-edge power distance from $x_0$ to $x_n$. Second, constructing a function
$\Conv$ satisfying the properties listed above.

When $q = 2$, we can do this for any $n$, where $x_0, \ldots x_n$ form a
$2$-screw simplex. This will prove Theorem~\ref{}. When $q > 2$, we can do this
for any four points that are vertices of a $q$-screw simplex. This will prove
Theorem~\ref{}. We further conjecture our approach will work for any $q > 2$
and $n$.

