\section{Fast, Sparse Spanner for the Edge-Squared
Metric}
\label{sec:general-spanner}

Now we outline a proof for Theorem~\ref{thm:general-spanner}, which shows that one can
construct a $(1+\eps)$ edge-squared spanner of size $O(n \eps^{-d/2})$ in
time $O\left(n \log n + n \eps^{-d/2} \log \left(\frac{1}{\eps}\right)\right)$, 
for points in constant dimensional space. The full proof is in
Appendix~\ref{ap:general-spanner}. By Theorem~\ref{thm:NN}, this
spanner is also a good spanner of the Nearest Neighbor Metric distance.
Note that this spanner is sparser and faster in terms of epsilon dependency than the best spanner for
Euclidean distances known to the authors
, which has $O\left(\eps^{-d}\right)$ edges and runs in
$O\left(n \log n +
\eps^{-d}n\log\left(\frac{1}{\eps}\right)\right)$
time~\cite{Callahan1993}. We rely extensively on well-separated pair decompositions
(WSPDs),
and this outline assumes familiarity with that notation.
For a comprehensive set of definitions and notations on well separated
pairs, refer to any of \cite{Callahan1995, Arya2016, Callahan1993,
arya95euclid}.  
Our proof consists of three parts.
\begin{enumerate}
\item Showing that connecting a $(1+O(\delta^2))$-approximate shortest edge
in a $1/\delta$ well separated pair for all the pairs in the decomposition
gives a $1+O(\delta^2)$ edge-squared spanner.
The processing for this step takes $O(n \log n + \delta^{-d}n)$ time.
\item Previous work contains an algorithm computing
  $1+O(\delta^2)$-approximate shortest edge in a $1/\delta$ well
    separated pair for all the pairs in a WSPD, and takes
    $O(1)$ time per pair. The pre-processing for this step will be
    bounded by $O(\delta^{-d}n\log\left(\frac{1}{\delta}\right))$ time. The $\log\left(\frac{1}{\delta}\right)$ factor goes away given a fast floor function. 
    This procedure was first introduced in~\cite{Callahan1995}.

\item Putting these two together, and setting $\epsilon = \delta^2$
gives us a $1+\epsilon$ spanner with
$O(\epsilon^{-d/2}n)$ edges in
    $O(n \log n + \epsilon^{-d/2}n)$ time.
\end{enumerate}
Full details of this proof are contained in Appendix~\ref{ap:general-spanner}

\subsection{Lower Bounds for Sparsity of Euclidean
Spanners}\label{sec:lower-bound}
\begin{theorem} \label{thm:euc} For constant $d$ and any fixed $\eps$, there exists a set
of points such that any $(1+\eps)$ Euclidean spanner in $\mathbb{R}^d$
  needs $\Omega\left(n
  \eps^{-\floor{d/2} + 1}\right)$ edges.
\end{theorem}
Here, we show that our edge-squared spanner is about as sparse as the
theoretically optimal Euclidean spanner with the same
approximation quality. 
The set of points is chosen adversarially for a given $\eps$.

\begin{proof} (of Theorem~\ref{thm:euc})
Take points spaced at least $4\epsilon$
apart on the surface of the unit ball on the first $d/2$ dimensions.
Then,
take points spaced at least $4\epsilon$ apart on the surface of the unit
ball on the remaining
$d/2$ dimensions. Let the first
set of points be $A$, and the second set of points be $B$.  
You can pack $\Theta(\eps^{-d/2+1})$ points into both $A$ and
$B$ this way.
Each distance crossing from $A$ to $B$ has Euclidean distance exactly equal to $2$.
Therefore, any edge from $A$ to $B$ must be in a
$(1+\eps)$ spanner of the Euclidean distance. We have constructed
a set $P := A \cup B$ with $\Theta(\eps^{-d/2+1})$ points, whose $(1+\eps)$ Euclidean spanner must have at
least $\Theta(\eps^{-d+2})$ edges.
This construction can have arbitrarily many points $n$, by
  duplicating $n \eps^{d/2-1}$
copies of $P$ arbitrarily far away from each other. The
  result has $n$ vertices, and must have at least $\Omega(n \eps^{-d/2+1})$ edges in
any $(1+\epsilon)$ Euclidean spanner.  
\end{proof}
By substituting $\sqrt{\eps}$ for $\eps$ in the construction, we
can additionally show a lower bound for the sparsity of an
edge-squared spanner.

\begin{lemma} For constant $d$ and any fixed $\eps$, there exists a point set
where a $(1+\eps)$
edge-squared spanner must have at least $\Omega\left(n\eps^{-\floor{d/4}+1}\right)$
edges. 
\end{lemma}
The point set is chosen adversarially for a given $\eps$.  
By setting $d=4$ and $\eps = \frac{1}{n}$, our construction gives:
\begin{lemma}\label{lem:n-squared}
There exists a 
$4$-dimensional set of points, such that any $1$-spanner of the
  edge-squared metric has $\Omega(n^2)$ 
edges.
\end{lemma}
