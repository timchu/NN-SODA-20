\subsection{Fast, Sparse Spanner for the Edge-Squared
Metric}
\label{sec:general-spanner}

Now we outline a proof for Theorem~\ref{thm:general-spanner}, which shows
that one can construct a $(1+\eps)$ nearest neighbor metric spanner of size
$O(n \eps^{-d/2})$ in
time $O\left(n \log n + n \eps^{-d/2} \log \left(\frac{1}{\eps}\right)\right)$, 
for points in constant dimensional space. The full proof is in
Appendix~\ref{ap:general-spanner}. We critically rely on
Theorem~\ref{thm:NN} for this work, which shows
edge-squared metric.

Note that this spanner is sparser and faster in terms of epsilon dependency
than the theoretical optimal spanner for Euclidean distances~\cite{Le19}.
We rely extensively on well-separated pair decompositions
(WSPDs),
and this outline assumes familiarity with that notation.
For a comprehensive set of definitions and notations on well separated
pairs, refer to any of \cite{Callahan1995, Arya2016, Callahan1993,
arya95euclid}.  
Our proof consists of three parts.
\begin{enumerate}
\item Showing that connecting a $(1+O(\delta^2))$-approximate shortest edge
in a $1/\delta$ well separated pair for all the pairs in the decomposition
gives a $1+O(\delta^2)$ edge-squared spanner.
The processing for this step takes $O(n \log n + \delta^{-d}n)$ time.
\item Previous work contains an algorithm computing
  $1+O(\delta^2)$-approximate shortest edge in a $1/\delta$ well
    separated pair for all the pairs in a WSPD, and takes
    $O(1)$ time per pair. The pre-processing for this step will be
    bounded by $O(\delta^{-d}n\log\left(\frac{1}{\delta}\right))$ time. The $\log\left(\frac{1}{\delta}\right)$ factor goes away given a fast floor function. 
    This procedure was first introduced in~\cite{Callahan1995}.

\item Putting these two together, and setting $\epsilon = \delta^2$
gives us a $1+\epsilon$ spanner with
$O(\epsilon^{-d/2}n)$ edges in
    $O(n \log n + \epsilon^{-d/2}n)$ time.
\end{enumerate}
Full details of this proof are contained in Appendix~\ref{ap:general-spanner}

\subsection{Lower Bounds for Sparsity of Euclidean
Spanners}\label{sec:lower-bound}
\begin{theorem} \label{thm:euc} For constant $d$ and any fixed $\eps$, there exists a set
of points such that any $(1+\eps)$ Euclidean spanner in $\mathbb{R}^d$
  needs $\Omega\left(n
  \eps^{-\floor{d/2} + 1}\right)$ edges.
\end{theorem}
