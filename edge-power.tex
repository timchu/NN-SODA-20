\section{Relating the Nearest Neighbor Metric to Euclidean MSTs,
Euclidean Spanners, and More}\label{sec:edge-power}
The Nearest Neighbor metric, as seen in Theorem~\ref{thm:NN}, is equal to
the edge-squared metric. This allows us to connect this manifold distance
to a graph distance, which we will in turn show is a generalization of
maximum-edge distance on minimum spanning trees.

The edge-squared metric on a Euclidean point set, as we recall, is defined by taking the
Euclidean distances squared and finding the shortest paths. 
We could have taken any such power $p$ of the Euclidean distances. We
will soon see that taking $p = 1$ gives us the Euclidean distance, and
finding spanners of the graph as $\lim p\rightarrow \infty$ is the
Euclidean MST problem.  
Let the $p$-power metric be defined on a Euclidean point set by taking
Euclidean distances to the power of $p$, and performing all-pairs
shortest path on the resulting distance graph.
\begin{theorem} \label{thm:1-spanner}
For all $q > p$, any $1$-spanner of the $p$-power metric is a
$1$-spanner of the $q$-power metric on the same point set
\end{theorem}
\begin{proof}
A $1$-spanner of the $q$-power metric can be made by taking
  edges $uv$ where
  \begin{align}\label{eq:span}
    \min_{p_0=u, \ldots p_k=v, k \not= 1} \sum_k ||p_i - p_{i-1}||^q > || u -
v||^q.
  \end{align}
If
  $\sum_{i=1}^k ||p_i - p_{i-1}||^q > || u - v||^q$ for any points
  $p_1, \ldots p_k$, then
  $\sum_{i=1}^k ||p_i - p_{i-1}||^p > || u - v||^p$ for any $q > p$.
  Thus, for all such edges $uv$ satisfying
  Equation~\ref{eq:span}:
  \[ \min_{p_0=u, \ldots p_k=v, k \not= 1} \sum_k ||p_i - p_{i-1}||^p > || u -
  v||^p. \] Such edges $uv$ must be included in any $1$-spanner
  of the $p$-power
  metric.
\end{proof}

 \begin{corollary}
   Let $P$ be a set of points in Euclidean space drawn i.i.d. from a Lipschitz
   probability density bounded above and below, with support on a
   smooth, compact manifold with intrinsic dimension $d$, bounded
   curvature, and
   smooth boundary of bounded curvature. Then the $k$-NN graph on $P$
   when $k = O(2^d \log n)$ is a $1$-spanner of the $p$-power
   metric for every $p \geq 2$, w.h.p.
 \end{corollary}
This follows from combining Theorem~\ref{thm:distribution-spanner} and
Theorem~\ref{thm:1-spanner}.
\subsection{Relation to the Euclidean MST problem}
  \begin{definition}
  Let the \textbf{normalized $p$-power metric} between two points in
  $\mathbb{R}^d$ be the $p$-power metric between the two points,
  raised to the $\frac{1}{p}$ power. Define the normalized $\infty$-power
  metric as the limit of the normalized $p$-power metric as $p \rightarrow \infty$.
  \end{definition}
  \begin{lemma} The Euclidean MST is a
  $1$-spanner for the normalized $\infty$-power metric.
  \end{lemma}
  This lemma follows from basic properties of the MST.
  The normalized $p$-power metrics give us a suite of
  metrics such that $p=1$ is the Euclidean
  distance and $p=\infty$ gives us the distance of the longest edge on the
  unique MST-path.  Setting $p=2$ gives the edge-squared metric, which
  sits between the Euclidean and max-edge-on-MST-path distance. 
  Theorem~\ref{thm:1-spanner} establishes
  that minimal $1$-spanners of the (normalized) $p$-power
  metric are contained in each other, as $p$ varies from
  $1$ to $\infty$. The minimal spanner for a general point set when $p=1$ is the complete graph, and
 the Euclidean MST is the minimal spanner for $p=\infty$. Thus:
  \begin{theorem} 
    For points in $\mathbb{R}^d$, every $1$-spanner of the $p$-power
    metric on that set of points contains every Euclidean MST.
  \end{theorem}
  \begin{corollary}
    Every $1$-spanner for the edge-squared metric and/or Nearest
    Neighbor Geodesic contains every Euclidean
    MST. 
  \end{corollary}
  \subsection{Generalizing Single Linkage Clustering, Level Sets, and k-Centers
  clustering}
  If our point set is drawn from a well-behaved probability
  density, then the normalized edge-power metrics
  converge to a nice geodesic distance detailed
  in~\cite{hwang2016}. When $p=1$, clustering with this metric is
  the same as Euclidean metric clustering ($k$-means,
  $k$-medians, $k$-centers), and when $p=\infty$, clustering with
  this metric is the same as the widely used level-set
  method~\cite{Wishart69, Gower1969, Ester1996,OPTICS96}. Thus, clustering
  with normalized edge-power metrics generalizes these two very
  popular methods, and interpolates between their advantages.
  Definitions of
  the level-set method and a full discussion are contained in
  Appendix~\ref{ap:clustering-link}

% \tim{Maybe you want to mention the self-sparsifying graph when you raise
% $p$ from $1$ to $\infty$}
