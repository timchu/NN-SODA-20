% !TeX root = main.tex

\subsection{Definitions and Preliminaries} % (fold)
\label{sec:definitions}
\noindent \textbf{q-power Nearest Neighbor Metric (q-NN metric):}
\noindent \textbf{q-edge power metric:}
\noindent \textbf{Unit Flow}
\noindent \textbf{Min-Cost Flow:}
\noindent \textbf{Barycentric Subdivision:}
\noindent \textbf{Fractional Laplacian:}
\noindent \textbf{Effective Resistance:}
\noindent \textbf{Negative Type Distance:}
\noindent \textbf{(Multiplicative) Spanner}
\noindent \textbf{Gabriel Graph:}
\noindent \textbf{Hwang and Hero Distance:}

\noindent \textbf{Edge-squared metric:}  For $x\in \R^d$, let $\|x\|$ denote the Euclidean norm.
  For a set of points $P\subset \R^d$:

  \begin{definition}
  The edge-squared metric for $a,b\in P$ is
  \[
    \dist_2(a,b) = \min_{(p_0,\ldots, p_k)}\sum_{i=1}^k \|p_i - p_{i-1}\|^2,
  \]
  where the minimum is over sequences of points $p_0,\ldots, p_k\in P$ with $p_0 = a$ and $p_k = b$.
  \end{definition}
%  \tim{Anything on wireless networks here? Or other prelims?}

\vspace{3 mm}
\noindent \textbf{Nearest-neighbor geodesic distance:}
  Another metric on the points of $P$ is called the nearest-neighbor
  geodesic distance, and is denoted $\dist_N$.
This distance was first defined and studied
  in~\cite{cohen15approximating}.
  Before we can define it, we need a couple other definitions.

  Given any finite set $P\subset \R^k$, there is a real-valued function $\distto_P: \R^k\to \R$ defined as $\distto_P(z) = \min_{x\in P} \|x-z\|$.
  A path is a continuous mapping $\gamma:[0,1]\to \R^d$.
  Let $\ourpath(a,b)$ denote the set of piecewise-$C_1$ paths from $a$ to $b$.
  We will compute the lengths of paths relative to the distance function $\distto_P$ as follows.
  \[
    \len(\gamma) := \int_0^1 \distto_P(\gamma(t))\|\gamma'(t)\|dt.
  \]
  By considering the velocity of $\gamma$, this definition is
  independent of the parameterization of the path.
  \begin{definition}
    The nearest-neighbor geodesic distance is defined as:
  \[
    \dist_N(a,b) := 4 \inf_{\gamma\in \ourpath(a,b)}\len(\gamma).
  \]
  The factor of $4$ normalizes the metrics.
  \end{definition}
  In particular, when $P$ has only two points $a$ and $b$, $\dist_2(a,b) = \dist_N(a,b)$.
  This reduces to a high school calculus exercise as the minimum path
  $\gamma$ will be a straight line between the points and the nearest
  neighbor geodesic is
  \[
    \dist_N(a,b) = 4\int_0^1 \distto_P(\gamma(t))\|\gamma'(t)\|dt = 8\int_0^{\frac{1}{2}} t \|a - b\|^2 dt = \|a - b\|^2 = \dist_2(a,b).
  \]

  This observation about pairs of points makes it easy to see that the
  nearest-neighbor geodesic distance is never greater than the
  edge-squared distance, as proven in the following lemma.

  \begin{lemma}\label{lem:dist_N_le_dist}
    For all $s,p\in P$, we have $\dist_N(s,p)\le \dist_2(s,p)$.
  \end{lemma}
  \begin{proof}
    Fix any points $s,p\in P$.
    Let $q_0,\ldots, q_k \in P$ be such that $q_0 = s$, $q_k = p$ and
    \[
      \dist_2(s,p) = \sum_{i=1}^k \|q_i - q_{i-1}\|^2.
    \]
    Let $\psi_i(t) = tq_i + (1-t)q_{i-1}$ be the straight line segment from $q_{i-1}$ to $q_i$.
    Observe that $\len(\psi_i) = \|q_i - q_{i-1}\|^2 / 4$, by the same argument as in the two point case.
    Then, let $\psi$ be the concatenation of the $\psi_i$ and it follows that
    \[
      \dist_2(s,p) = 4 \len(\psi) \ge 4 \inf_{\gamma\in \ourpath(s,p)} \len(\gamma) = \dist_N(s,p).\qedhere
    \]
  \end{proof}

\vspace{3 mm}
  
\noindent \textbf{Spanners:} For real value $t \geq 1$, a $t$-spanner of
a weighted graph $G$ is a subgraph $S$ such that $d_G(x,y) \leq d_S(x,y)
\leq t\cdot d_G(x,y)$ where $d_G$ and $d_S$ represent the shortest path
distance functions between vertex pairs in $G$ and $S$. Spanners
of Euclidean distances, and general graph distances, have been
studied extensively, and their importance as a data structure is
well established.
~\cite{Chew1986, Vaidya1991, Callahan1993,HarPeled13}.

\vspace{3 mm}
\noindent \textbf{$k$-nearest neighbor graphs:} The $k$-nearest neighbor graph
($k$-NN graph) for a set of objects $V$ is a graph with vertex set $V$
and an edge from $v\in V$ to its $k$ most similar objects in $V$, under
a given distance measure. In this paper, the underlying distance
measure is Euclidean, and the edge weights are Euclidean distance
squared.
$k$-NN
graph constructions are a key data structure in machine
learning~\cite{Dong11, Chen11}, clustering~\cite{vL09}, and manifold learning~\cite{tenenbaum00global}.

\vspace{3 mm}
\noindent \textbf{Gabriel Graphs:} The Gabriel graph is a graph where
two vertices $p$ and $q$ are joined by an edge if and only if the disk
with diameter $pq$ has no other points of $S$ in the interior. The
Gabriel graph is a subgraph of the Delaunay
triangulation~\cite{SridharMaster}, and a
$1$-spanner of the edge-squared metric~\cite{SridharMaster}. Gabriel
graphs will be used in the proof of
Theorem~\ref{thm:distribution-spanner}.
% section definitions (end)
