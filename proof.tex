%\subsection{Equivalence} % (fold)
In this section, we prove Theorem~\ref{thm:NN} on finite point sets, and
explain in Section~\ref{sec:bodies} that our proof strategy applies to
finite collections of path-connected compact bodies.
\label{sec:the_proof}

  Let $P\subset \R^d$ be a set of $n$ points.
  Pick any \emph{source} point $s\in P$.
  Order the points of $P$ as $p_1,\ldots ,p_n$ so that
  \[
    \dist_2(s,p_1) \le \cdots \le \dist_2(s, p_n).
  \]
  This will imply that $p_1 = s$.
  It will suffice to show that for all $p_i\in P$, we have $\dist_2(s,p_i) = \dist_N(s,p_i)$.
  There are three main steps:
  \begin{enumerate}
    \item We first show that when $P$ is a subset of the vertices of an axis-aligned box, $\dist = \dist_N$.  In this case, shortest paths for $\dist$ are single edges and shortest paths for $\dist_N$ are straight lines.
    \item We then show how to lift the points from $\R^d$ to $\R^n$ by a Lipschitz map $m$ that places all the points on the vertices of a box and preserves $\dist_2(s,p)$ for all $p\in P$.
    \item Finally, we show how the Lipschitz extension of $m$ is also Lipschitz as a function between Nearest Neighbor Metric distances.  We combine these pieces to show that $\dist \le \dist_N$.  As $\dist \ge \dist_N$ (Lemma~\ref{lem:dist_N_le_dist}), this will conclude the proof that $\dist = \dist_N$.
  \end{enumerate}
  % !TeX root = main.tex

\subsubsection{Boxes} % (fold)
\label{sec:boxes}

  Let $Q$ be the vertices of a box in $\R^n$.
  That is, there exist some positive real numbers $\alpha_1,\ldots , \alpha_n$ such that each $q\in Q$ can be written as $q = \sum_{i\in I} \alpha_i e_i$, for some $I\subseteq [n]$.

  Let the source $s$ be the origin.
  Let $\distto_Q:\R^n\to \R$ be the distance function to the set $Q$.
  Setting $r_i(x) := \min\{x_i, \alpha_i - x_i\}$ (a lower bound on the difference in the $i$th coordinate to a vertex of the box), it follows that
  \begin{equation}
    \label{eq:distto_bded_by_ris}
    \distto_Q(x) \ge \sqrt{\sum_{i= 1}^n r_i(x)^2}.
  \end{equation}

  Let $\gamma:[0,1]\to \R^n$ be a curve in $\R^n$.
  Define $\gamma_i(t)$ to be the projection of $\gamma$ onto its $i$th coordinate.
  Thus,
  \begin{equation}\label{eq:ri_as_min}
    r_i(\gamma(t)) = \min\{\gamma_i(t), \alpha_i - \gamma_i(t)\}
  \end{equation}
  and
  \begin{equation}\label{eq:gamma_decomposed}
    \|\gamma'(t)\| = \sqrt{\sum_{i = 1}^n \gamma_i'(t)^2}.
  \end{equation}
  %
  We can bound the length of $\gamma$ as follows.
  %
  \begin{align*}
    \len(\gamma)
      &= \int_0^1 \distto_Q(\gamma(t))\|\gamma'(t)\|dt \because{by definition}\\
      &\ge \int_0^1 \left(\sqrt{\sum_{i= 1}^n r_i(\gamma(t))^2} \sqrt{\sum_{i = 1}^n \gamma_i'(t)^2}\right) dt \because{by \eqref{eq:distto_bded_by_ris} and \eqref{eq:gamma_decomposed}}\\
      &\ge \sum_{i=1}^n \int_0^1 r_i(\gamma(t)) \gamma_i'(t) dt \because{Cauchy-Schwarz}\\
      &= \sum_{i=1}^n \left(\int_0^{\ell_i} \gamma_i(t) \gamma_i'(t)dt + \int_{\ell_i}^1 (\alpha_i - \gamma_i(t)) \gamma_i'(t) dt\right) \because{by \eqref{eq:ri_as_min} where $\gamma_i(\ell_i) = \alpha_i/2$}\\
      &= \sum_{i=1}^n 2\int_0^{\ell_i} \gamma_i(t) \gamma_i'(t) dt \because{by symmetry}\\
      &= \sum_{i=1}^n \frac{\alpha_i^2}{4} \because{basic calculus}
  \end{align*}

  It follows that if $\gamma$ is any curve that starts at $s$ and ends at $p = \sum_{i=1}^n \alpha_i e_i$, then $\dist_N(s,p) = \dist_2(s,p)$.

% section boxes (end)

  \subsubsection{Generating a Lipschitz embedding into \R^n} % (fold)
\label{sec:lifting}
  
We seek to find points $m(p_i) \in \R^n$ such that 
\begin{align}\label{eq:preserve}
\dist_q(m(s,p_i),m(s,p_{i-1})) = \dist_q(s, p_i) - \dist_q(s,p_{i-1} )$
\end{align}

To find $m$, we perform a breadth-first search to find points on the real
line $x_0 < x_1 < \ldots x_{n-1}$ such that $x_i-x_{i-1} = \dist_q(s, p_i
- dist_q(s, p_{i-1})$. These points $x_i$ can be found with a simple
  breadth first search on our points. Note that if we set $m(p_0), \ldots
m(p_{n-1})$ as the vertices of the $q$-screw simplex
formed from points $x_0, x_1, \ldots x_n$, then Equation~\ref{eq:preserve}
holds.


% section lifting (end)

  \subsubsection{The Lipschitz Extension} % (fold)
\label{sec:lip_extension}

  Proposition~\ref{prop:m_is_good} and the Kirszbraun theorem on Lipschitz extensions imply that we can extend $m$ to a $1$-Lipschitz function $f: \R^d\to \R^n$ such that $f(p) = m(p)$ for all $p\in P$ \cite{Kirszbraun1934,Valentine1945,brehm1981}.

  \begin{lemma}\label{lem:dist_N_lipschitz}
    The function $f$ is also $1$-Lipschitz as mapping from $\R^d\to \R^n$ with both spaces endowed with the nearest neighbor geodesic.
  \end{lemma}
  \begin{proof}
    We are interested in two distance functions $\distto_P:\R^d \to \R$ and $\distto_{f(P)}: \R^n\to \R$.
    Recall that each is the distance to the nearest point in $P$ or $f(P)$ respectively.
    \begin{align*}
      \distto_{f(P)}(f(x)) 
        &= \min_{q\in f(P)} \|q - f(x)\| \because{by definition}\\
        &= \min_{p\in P} \|f(p) - f(x)\| \because{$q = f(p)$ for some $p$}\\
        &\le \min_{p\in P} \|p - x\| \because{$f$ is $1$-Lipschitz}\\ 
        &= \distto_P(x). \because{by definition}
    \end{align*}
    For any curve $\gamma:[0,1]\to \R^d$ and for all $t\in [0,1]$, we have $\|(f\circ \gamma)'(t)\| \le \|\gamma'(t)\|$.
    It then follows that
    \begin{equation}\label{eq:curves_shorten}
      \len'(f\circ \gamma) = \int_0^1 \distto_{f(P)}(f(\gamma(t)))\|(f\circ\gamma)'(t)\|dt \le \int_0^1 \distto_{P}(\gamma(t))\|\gamma'(t)\|dt = \len(\gamma),
    \end{equation}
    where $\len'$ denotes the length with respect to $\distto_{f(P)}$.
    Thus, for all $a,b\in P$,
    \begin{align*}
      \dist_N(a,b)
        &= 4 \inf_{\gamma\in \ourpath(a,b)} \len(\gamma) \because{by definition}\\
        &\ge 4 \inf_{\gamma\in \ourpath(a,b)} \len'(f\circ\gamma) \because{by \eqref{eq:curves_shorten}}\\
        &\ge 4 \inf_{\gamma'\in \ourpath(f(a),f(b))} \len'(\gamma') \because{because $f\circ\gamma\in \ourpath(f(a), f(b))$}\\
        &= \dist_N(f(a), f(b)). \because{by definition}
    \end{align*}
  \end{proof}
  
  \begin{theorem}\label{thm:equality}
    For any point set $P\subset\R^d$, the edge squared metric $\dist$ and the nearest neighbor geodesic $\dist_N$ are identical.
  \end{theorem}
  \begin{proof}
    Fix any pair of points $s$ and $p$ in $P$.
    Define the Lipschitz mapping $m$ and its extension $f$ as in \eqref{eq:defn_of_m}.
    Let $\dist'$ and $\dist_N'$ denote the edge-squared and nearest neighbor geodesics on $f(P)$ in $\R^n$.
    \begin{align*}
      \dist_2(s,p) 
        &= \dist'(m(s), m(p)) \because{Proposition~\ref{prop:m_is_good}(iii)}\\
        &= \dist_N'(m(s), m(p)) \because{$f(P)$ are vertices of a box}\\
        &\le \dist_N(s, p) \because{Lemma~\ref{lem:dist_N_lipschitz}}
    \end{align*}
    We have just shown that $\dist\le \dist_N$ and Lemma~\ref{lem:dist_N_le_dist} states that $\dist\ge \dist_N$, so we conclude that $\dist = \dist_N$ as desired.
  \end{proof}

% section lip_extension (end)

\subsection{From Finite Sets to Finite Collections of Compact Path-Connected Bodies}
All of our proof steps hold for finite collections of compact,
path-connected bodies in arbitrarily large dimension. Our Lipschitz map $m$ can
still be extended to a map $f$ in this setting, largely due to the
generality of the Kirszbraun theorem. Moreover, $m$ contracts each convex
body into a single point, by our construction. Therefore, the image of our
compact bodies under $f$ is still a finite point set on the corners of a
box, and the remainder of our theorem proof goes through.

This result is rather remarkable: path-connected compact sets in high
dimensional space can have extremely convoluted geometry, and the Voronoi diagrams
on these collections (on which the nearest neighbor metric depends) can be
massively complex.  The key is htat our Lipschitz map is robust enough to
handle objects of considerable geometric complexity.
  
  
% section the_proof (end)
