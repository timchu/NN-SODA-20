\subsubsection{Lifting the points to $\R^n$} % (fold)
\label{sec:lifting}
  
  Define a mapping $m: P \to \R^n$ so that $m(p_1) = 0$ and otherwise 
  \begin{equation}\label{eq:defn_of_m}
    m(p_i) = m(p_{i-1}) + \left(\dist_q(s, p_i) - \dist_q(s,
p_{i-1})\right)^{1/q} e_i,
  \end{equation}
  where the vectors $e_i$ are the standard basis vectors in $\R^n$.

  \begin{lemma}\label{lem:m_and_dist}
    For all $p_i, p_j\in P$, we have 
    \begin{enumerate}
      \item[(i)] $\|m(p_j) - m(p_i)\| = \sqrt{|\dist_q(s,p_j) - \dist_q(s,p_i)|}$, and
      \item[(ii)]$\|m(s) - m(p_j)\|^2 \le \|m(p_i)\|^2 + \|m(p_i) - m(p_j)\|^2$.
    \end{enumerate}
  \end{lemma}
  \begin{proof}
    \emph{Proof of (i).}
    Without loss of generality, let $i \le j$.
    \begin{align*}
      \|m(p_j) - m(p_i)\| &= \left\|\sum_{k=i+1}^j \sqrt{\dist_q(s, p_k) - \dist_q(s, p_{k-1})} e_k \right\| \because{from the definition of $m$}\\
      &= \sqrt{\sum_{k=i+1}^j (\dist_q(s, p_k) - \dist_q(s, p_{k-1}))}\because{expand the norm}\\
      &= \sqrt{\dist_q(s, p_j) - \dist_q(s, p_i)}. \because{telescope the sum}
    \end{align*}

    \noindent\emph{Proof of (ii).}
    As $m(s) = 0$, it suffice to observe that
    \begin{align*}
      \|m(p_j)\|^q
        &= \dist_q(s, p_j) \because{by \emph{(i)}}\\
        &\le \dist_q(s, p_i) + |\dist_q(s,p_j) - \dist_q(s,p_i)|\because{basic arithmetic}\\
        &= \|m_(p_i)\|^q + \|m(p_i) - m(p_j)\|^q \because{by \emph{(i)}}
    \end{align*}
  \end{proof}

  We can now show that $m$ has all of the desired properties.

  \begin{prop}\label{prop:m_is_good}
    Let $P\subset\R^d$ be a set of $n$ points, let $s\in P$ be a designated source point, and let $m:P\to \R^n$ be the map defined as in \eqref{eq:defn_of_m}.
    Let $\dist'$ denote the edge squared metric for the point set $m(P)$ in $\R^n$.
    Then,
    \begin{enumerate}
      \item[(i)] $m$ is $1$-Lipschitz as a map between Euclidean metrics,
      \item[(ii)] $m$ maps the points of $P$ to the vertices of a box, and
      \item[(iii)] $m$ preserves the edge squared distance to $s$, i.e.\ $\dist'(m(s), m(p)) = \dist_q(s,p)$ for all $p\in P$.
    \end{enumerate} 
  \end{prop}
  \begin{proof}
    \emph{Proof of (i).} To prove the Lipschitz condition, fix any $a,b\in P$ and bound the distance as follows.
    \begin{align*}
      \|m(a) - m(b)\| 
        &= \sqrt{|\dist_q(s,a) - \dist_q(s,b)|} \because{Lemma~\ref{lem:m_and_dist}(i)}\\
        &\le \sqrt{\dist_q(a,b)} \because{triangle inequality}\\
        &\le \|a-b\| \because{$\dist_q(a,b)\le \|a-b\|^q$ by the definition of $\dist$}
    \end{align*}

    \noindent
    \emph{Proof of (ii).} That $m$ maps $P$ to the vertices of a box is immediate from the definition.
    The box has side lengths $\|m_i - m_{i-1}\|$ for all $i>1$ and $p_i = \sum_{k=1}^i \|m_k - m_{k-1}\| e_k$.

    \noindent
    \emph{Proof of (iii).} We can now show that the edge squared distance to $s$ is preserved.
    Let $q_0,\ldots, q_k$ be the shortest sequence of points of $m(P)$ that realizes the edge-squared distance from $m(s)$ to $m(p)$, i.e., $q_0 = m(s)$, $q_k = m(p)$, and 
    \[
      \dist'(m(s), m(p)) = \sum_{i = 1}^k \|m(q_i) - m(q_{i-1})\|^2.
    \]
    If $k> 1$, then Lemma~\ref{lem:m_and_dist}(ii) implies that removing $q_1$ gives a shorter sequence.
    Thus, we may assume $k = 1$ and therefore, by Lemma~\ref{lem:m_and_dist}(i),
    \[
      \dist'(m(s), m(p)) = \|m(s) - m(p)\|^2 = \dist_q(s,p).\qedhere
    \]
  \end{proof}
  

% section lifting (end)
