\section{Outline}
%   Definitions of the nearest-neighbor geodesic distance, and of the
%     edge-squared metric, are provided in
%    Section~\ref{sec:definitions}.
%  
Section~\ref{sec:metric-equality} contains the proof of Theorem~\ref{thm:NN}
and~\ref{thm:qNN}. This section contains most of the novel mathematical
ideas in our work, including how to relate a manifold distance to min-cost
flows on a simplex. It also establishes the centrality of the $q$-screw
simplex in our proof techniques.

Section~\ref{sec:isometry} contains the link from $q$-screw simplices to
fractional Laplacians and spectral graph theory, as well as the proof for
Theorems~\ref{thm:ER} and~\ref{thm:l1}.

Section~\ref{sec:relation-to-MST} shows a brief proof that the Nearest Neighbor
metric, and the $q$-edge power metrics in general, generalize common
metrics like maximum-edge Euclidean MST metrics and Euclidean distance. We
further show how clustering algorithms using $q$-edge poewr metrics
generalize $k$-means, level-set methods, single linkage clustering, and
more.

Persistent homology of the Nearest Neighbor metric is contained in
Sectoin~\ref{sec:homology}.

Section~\ref{sec:spanner} contains overviews of the proof for both
Theorem~\ref{thm:general-spanner} and~\ref{thm:distribution-spanner}, as
well as their implications. Full
proofs are contained in the Appendix.

Section~\ref{sec:limit-behavior} shows the convergence of the Nearest Neighbor
metic, and the $q$-edge power metrics in general, to a beautiful geodesic
on an underlying probability distribution previously studied by Hwang et.
al.

Section~\ref{sec:conclusion} contains conclusions, a summary of work, and
open questions for the future.
