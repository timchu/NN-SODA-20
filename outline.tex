\section{Outline}
%   Definitions of the Nearest Neighbor Metric distance, and of the
%     edge-squared metric, are provided in
%    Section~\ref{sec:definitions}.
%  
Section~\ref{sec:proof} contains the proof of Theorem~\ref{thm:NN},
equating the edge-squared metric and Nearest Neighbor Metric distance in
all cases. It should be noted that our proof is robust enough to handle not
just finite point sets, but also countably infinite collections of disjoint
path-connected, compact sets. Remarkably, there is no restriction on the convexity or
simply-connectedness of these sets.

We then compute the persistent homology of the Nearest Neighbor Metric
distance.  Section~\ref{sec:general-spanner} outlines a proof of
Theorem~\ref{thm:general-spanner}, and compares our spanner to new lower
bounds
on the sparsity of $(1+\eps)$-spanners of the Euclidean metric.
We outline a proof of Theorem~\ref{thm:distribution-spanner} in
Section~\ref{sec:distribution-spanner} and discuss its implications.  

Section~\ref{sec:edge-power} introduces the $p$-power metrics. We show
that Euclidean spanners and Euclidean MSTs are special cases of
$p$-power spanners. We show how
clustering algorithms including $k$-means, level-set methods,
and single linkage clustering, are special cases of
clustering with $p$-power metrics.


Conclusions and open questions are in
Section~\ref{sec:conclusions}. Full proofs for
Theorems~\ref{thm:distribution-spanner},~\ref{thm:general-spanner}
are contained in the Appendix.
