\subsection{Contributions and Past Work}
Our primary contribution is Theorem~\ref{thm:NN}, which lets us exactly
compute the nearest neighbor metric. This significantly strengthens a core
result of Cohen et al~\cite{cohen15approximating}. This theorem should be considered
quite surprising: it equates the nearest neighbor metric with the
edge-squared metric, even when the point set is a collections of compact,
path-connected objects in arbitrarily large dimension. There are no
restrictions on the convexity or simple-connectedness of such objects, so
in general the Voronoi diagram of these objects (on which the nearest
neighbor metric critically depends) can be extremely complicated.

 Besides for exactly computing the nearest neighbor metric, we present the
following theorems on approximate computation:

 %% \begin{theorem} \label{thm:NN} Given a point set $P \in \mathbb{R}^d$, the edge-squared metric on $P$
%%   and the Nearest Neighbor Metric on $P$ are always equivalent.
%% \end{theorem}

\begin{theorem} \label{thm:general-spanner}
  For any set of points in $\mathbb{R}^d$ for constant $d$, there exists a $(1+\eps)$
  spanner of the Nearest Neighbor Metric
  with size $O\left(n\eps^{-d/2} \right)$ computable in time
  $O\left(n \log n + n\eps^{-d/2}\log{\frac{1}{\eps}}\right)$. The
  $\log{\frac{1}{\eps}}$ term goes away given access to an algorithm
computing floor function
in $O(1)$ time.
\end{theorem}

\begin{theorem} \label{thm:distribution-spanner}
Suppose points $P$ in Euclidean space are drawn i.i.d from a Lipschitz probability density bounded
above and below by a constant, with support on a
smooth, connected, compact manifold with intrinsic dimension $d$,
  and smooth
  boundary of bounded curvature. Then w.h.p. the $k$-NN graph of
  $P$ for $k = O(2^d \ln n)$ and edges weighted with Euclidean
  distance squared, is a $1$-spanner of the Nearest Neighbor
  metric on $P$.
\end{theorem}

These theorems rely on Theorem~\ref{thm:NN} and considerably strengthen the
spanner results on the nearest neighbor metric
from~\cite{cohen15approximating}. They critically rely on
Theorem~\ref{thm:NN}, which show it suffices to compute spanners of the
edge-squared metric.
Previously, sparse spanners of the edge-squared metric were shown to exist in two
dimensions via Yao graphs and Gabriel graphs~\cite{LiWan2001}, but these
did not generalize well to constant dimension: Yao
graphs are not very efficient to compute, and Gabriel graphs can have
quadratically many edges even in $3$ dimensions~\cite{chazelle94selecting}.

Theorem~\ref{thm:distribution-spanner} proves that a $1$-spanner of
the nearest neighbor metric can be found assuming points are samples from a
probability density, by using a $k$-$NN$ graph for
appropriate $k$. Our result is tight when $d$ is constant. This
is not possible for Euclidean distance, as a $1$-spanner is almost
surely the complete graph. Although the restrictions on the probability
density may seem limiting,
they are in fact quite flexible and standard in
machine learning theory and practice~\cite{alamgir12shortest, hwang2016}. For example, although they do not cover the case
of a Gaussian (unbounded support), they do cover the case of a Gaussian
where the very thin tail is cut off, and this recovers most of the relevant
data in a Gaussian distribution. Past work on similar results
include~\cite{Balister05, Gonzales2003}.

Finally, we show how the nearest neighbor metric generalizes Euclidean
distance and maximum-edge Euclidean MST distance ~\cite{LiWan2001}
Theorem~\ref{thm:NN} will also allow us
to compute the persistent homology of $\dist_N$, a task useful for
topological data analysis~\cite{edelsbrunner02topological}.

The core mathematical contribution of our work is the
statement and proof of Theorem~\ref{thm:NN}.  The techniques to prove
our other results are simpler and mostly leverage past work. We have
included them
nonetheless to provide a more complete picture of the nearest neighbor
metric and possible directions for future work.

%% Old outline: check this outline before submitting.
%% \begin{enumerate}
%% \item Definition of edge-squared.
%% \item Preliminaries
%% \item Outline/Overview/Previous Work
%% \item Interpretations based on known Machine Learning tools.
%% \begin{enumerate}
%% \item Gaussian Kernel similarity.
%% \item Generalization of Level-Set and Single-Linkage clustering. (\tim{This
%%     is not a very good point.})
%% \item Interpretation as $l_2$ on paths?
%% \end{enumerate}
%% \item Equality to a natural geodesic distance.
%% \begin{enumerate}
%% \item Define NN-metric.
%% \item Core Proof.
%% \item NN has a fast sparse spanner. Any spanner of the NN metric is a
%% spanner of edge-squared. (This result is theoretically
%%     superseded by our later result, but is of independent interest).
%% \item Persistent homology of the nearest neighbor metric can be
%% computed.
%% \end{enumerate}
%% \item Fast practical spanners for points in a distribution. ($k$-NN
%%     graph for $k = O( \log n)$.
%% \item Theoretically sparse, fast spanners for points in low dimension.  \tim{Low intrinsic dimension? Can cover trees do this for me?}
%% \item Remaining open questions.
%% \item Appendix:
%% \begin{enumerate}
%% \item Persistent homology can be put here, maybe.
%% \item Links to Heirarchies of Metrics: Edge-Squared is a natural
%% extension of negative type metrics, with potentially high doubling dimension.
%% \end{enumerate}
%%
%% \end{enumerate}
