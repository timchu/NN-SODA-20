\section{Our Results}
Our primary results are Theorems~\ref{thm:NN} and~\ref{thm:qNN} on the
computing the Nearest Neighbor and $q$-Nearest Neighbor metrics
respectively, which rely
on fairly intricate and novel mathematical proof techniques. Both of these
use a minimum cost flow generated from a conservative vector field to keep
track of the $q$-NN path length, and rely on the geometry of a special
simplex called the
$q$-screw simplex. Our proof techniques, theoretically, can prove
approximations between metrics as well as exact equalities, and may be
uesful for other more general metrics based on data points. This is the
first result we know of whose proof combines
conservative vector fields, simplex geometry, Lipschitz extensions in
geometry, and minimum cost flows on a graph. It is also the first result we
know of that eschews calculus of variations for exact manifold metric
computation.

These results form the core mathemtaical contribution of our paper.

\subsection{Other Contributions}
Besides for these contributions, we also tackle a range of problems
significant for any metric on a data set.  Some of the most important
problems on any metric are: how to $(1+\eps)$ approximate them efficiently
using sparse data structures, and how the metric behaves in the limit as
the point set is a large number of points drawn from a probability
distribution. The former is important to compute the metrics in practice.
The latter is important since it is highly desirable that
this limit converge to a metric that has desirable properties, or else
clustering with such a metric may generate non-intelligible results for
large datasets.  Besides for these, persistent homology of metrics space is also a central
tool in topological data analysis, among other fields~\cite{}, and many
papers have devoted themselves to computing such homologies ofr a wide
variety of metrics. Finally, it is important to relate metrics
like the Nearest Neighbor metric to famous, well-established metrics like
$l_2$, inverse min-cut distance, or
the maximum edge length among a path in an MST.
We will show that the
Nearest Neighbor metric is in fact one of a suite of simple metrics (the
$q$-edge power metrics) which naturally generalize both maximum-edge MST
distance and Euclidean distance, the latter of which covers inverse min-cut
idstances via the Gomory Hu tree.  This helps put our metrics in a broader
and more interpretable context, and ensures that clustering with these
$q$-power metrics is a generalization of more traditional clustering
techniques including $k$-means, $k$-centers, and level set methods.

In this paper, we present novel results all of these problems. Our
spanner and convergence results hold assuming constan dimension, and our
persistent homology results and our relation to more famous metrics hold for
$n$ points in any dimension. The proofs of these results are mostly
simpler than our proofs of Theorem~\ref{thm:NN}, and thus should be
appreciated mostly for the result statement rather than the complexity or
intricacy of their proofs. They all hinge on Theorem~\ref{thm:NN}, and are
largely included to show that
density-sensitive manifold metrics, like the NN metric, can be very
flexible when Theorems like Theorem~\ref{thm:NN} hold.

Our spanner theorems are as follows:

 %% \begin{theorem} \label{thm:NN} Given a point set $P \in \mathbb{R}^d$, the edge-squared metric on $P$
%%   and the Nearest Neighbor Metric on $P$ are always equivalent.
%% \end{theorem}

\begin{theorem} \label{thm:general-spanner}
  For any set of points in $\mathbb{R}^d$ for constant $d$, there exists a $(1+\eps)$
  spanner of the Nearest Neighbor Metric
  with size $O\left(n\eps^{-d/2} \right)$ computable in time
  $O\left(n \log n + n\eps^{-d/2}\log{\frac{1}{\eps}}\right)$. The
  $\log{\frac{1}{\eps}}$ term goes away given access to an algorithm
computing floor function
in $O(1)$ time.
\end{theorem}

\begin{theorem} \label{thm:distribution-spanner}
Suppose points $P$ in Euclidean space are drawn i.i.d from a Lipschitz probability density bounded
above and below by a constant, with support on a
smooth, connected, compact manifold with intrinsic dimension $d$,
  and smooth
  boundary of bounded curvature. Then w.h.p. the $k$-NN graph of
  $P$ for $k = O(2^d \ln n)$ and edges weighted with Euclidean
  distance squared, is a $1$-spanner of the Nearest Neighbor
  metric on $P$.
\end{theorem}


Theorem~\ref{thm:ditsribution-spanner} tackles a common setting in machine
learning, where points are assumed to be from a well-behaved ditsribution.
These assumption is foundational to the field of machine learning.
Although the restrictions on the distribution seem fairly limiting (and
naively, do not even cover the case of a simple Gaussian), they turn out to
be far more flexible than they seem, and are common in the machine learning
literature. They can be modified to gain information on most relevant
distributions (for example, they cover the
case of a Gaussian where the thin tail is removed, which turns out to
contain most of the information of a Gaussian).
Theorem~\ref{thm:general-spanner} generalizes the results of Cohen et. al.
in~\cite{cohen15approximating}, who showed nearly linear size spanners in
nearly linear time, but with significantly worse $\eps$ dependence. Note
that these spanners can be computed even faster than the optimal known
Euclidean spanner, indicating that density-sensitive metrics like the
Nearest Neighbor metric may have interesting algorithmic possibilities that
standard distances like $l_2$ and $l_1$ don't have.

Results on the geodesics, etc. will be presented in their appropriate
section.

