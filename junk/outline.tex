\section{Outline}
%   Definitions of the nearest-neighbor geodesic distance, and of the
%     edge-squared metric, are provided in
%    Section~\ref{sec:definitions}.
%  
Section~\ref{sec:metric-equality} contains the proof of Theorem~\ref{thm:NN}
and~\ref{thm:qNN}. This section contains most of the novel mathematical
ideas in our work, including how to relate a manifold distance to min-cost
flows on a simplex. It also establishes the centrality of the $q$-screw
simplex in our proof techniques.

Section~\ref{sec:isometry} shows how $q$-screw simplices relate to
fractional Laplacians and spectral graph theory, which allows us to prove
Theorem~\ref{thm:ER}. It also contains a new embedding of the $q$-screw
simplex for $q>2$, proving Theorem~\ref{thm:l1}. These proofs and results
are mostly included for independent interest, and to spur future work
relating $q$-screw simplices to density-sensitive distances.

We show a brief proof in Section~\ref{sec:relation-to-MST} that the Nearest
Neighbor metric, and the $q$-edge power metrics in general, generalize common
metrics like maximum-edge Euclidean MST metrics and Euclidean distance.
This will show how clustering algorithms using $q$-edge poewr metrics
generalize $k$-means, level-set methods, single linkage clustering, and
more.

Persistent homology of the Nearest Neighbor metric is contained in
Sectoin~\ref{sec:homology}.

Section~\ref{sec:spanner} contains overviews of the proof for both
Theorem~\ref{thm:general-spanner} and~\ref{thm:distribution-spanner}, as
well as their implications. Full
proofs are contained in the Appendix. These results are not particularly
tricky to prove, but they nonetheless solve important questions of fast
$(1+\eps)$ approximations of the Nearest Neighbor metric.

Section~\ref{sec:limit-behavior} shows the convergence of the Nearest Neighbor
metic, and the $q$-edge power metrics in general, to a beautiful geodesic
on an underlying probability distribution previously studied by Hwang et.
al. Most of the heavy lifting here was done by past researchers, such as
Hwang et. al. and Steele~\ref{}. However, this result lets us interpret
Nearest Neighbor metric clustering as an approximation of a very nice
clustering on an underlying probability density function.

Section~\ref{sec:conclusion} contains conclusions, a summary of work, and
open questions for the future.
