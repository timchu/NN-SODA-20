
Our final set of theorems regards the $q$-screw simplex, the core geometric
object in our proof of Theorem~\ref{thm:NN} and its generalizations. The
$q$-screw simplex was first discovered by John Von Neumann and Issai
Schoenberg. It is defined by taking $n$ points anywhere on a line, taking
the $1/q$ power of the distances, and isometrically embedding the resulting
distances into Euclidean space. The fact that such an embedding exists was
the core contribution of Schoenberg and Von Neumann in~\ref{}. The central
role of $q$-screw simplices in our proofs motivates us to develop new
theorems on the geometry of these objects. Surprisingly, we find that these
simplices are useful for proving generalizations of Von Neumann's work, and
are deeply related to spectral graph theory.

Isometric embedding is a topic of wide interest in the field of metric
geometry, and has been studied for many decades. Von Neumann and Issai
Schoenberg proved in their seminal work that any $q$-screw simplex is
isometrically embeddable in $l_2$. We extend their work to prove a stronger
result: 
\begin{theorem}
Any $q$-screw simplex isometrically embeds into the space of
Effective Resistance metrics, for all $q>1$.
\end{theorem}
Simple metrics like the square in $l_2$ are
not isometrically embeddable into this class of metrics, and thus, most
Euclidean metrics are not expected to isometrically embed into Effective
Resistance distance. Isometric embedding into effective resistance metrics
has been a popular question in spectral graph theory ~\cite{}, and this is
the first result we know of where a geometric distance defined without an
obvious underlying electrical network embeds isometrically into Effective
Resistances. We prove this isometry by showing a deep link between
$q$-screw simplices and a differential operator known as the fractional
Laplacian, which has wide applications in fields including fractional
quantum physics~\cite{}, cell membrane biology~\cite{}, financial
mathematics~\cite{}, Brownian motion~\cite{}, differential
equations~\cite{}, semi-groups~\cite{}, Fourier analysis~\cite{}, and
more~\cite{}. This further allows us to show that fundamental geometric
quantities like circumcenters (essential for Voronoi diagram construction)
and volumes on the $q$-screw simplex can be determined using fundamental
primitives on graph Laplacians, in this case Laplacian system solving and
Laplacian determinant estimation respectively. The fractional Laplacian can
be interpreted as a natural example of a geometric resistive graph, first
introduced by Alman et. al. in ~\cite{}. We further conjecture that taking
the $q^{th}$ root of any tree metric is isometrically embeddable into
effective resistance, which would imply that the Gomory Hu tree (and thus
the inverse min-cut distance) embeds isometrically into effective
resistances.

We also provide the first known closed form finite-dimensional embedding of
the $q$-screw simplex into Euclidean space, when $q>2$. The work of Von
Neumann et. Al. proved the simplex's existence for $q>1$ by embedding it
into infinite dimensional Hilbert space using theorems from complex

% Contributions
\subsection{Properties of the $q$-screw simplex, and related results}
Our secondary results are Theorems~\ref{thm:ER} and~\ref{thm:l1}, which are
isometric embeding results that emerge from studying the geometry of the
$q$-screw simplex.  These results are of independent interest from the main
text on density-sensitive distances, but are included since the $q$-screw
simplex is the fundamental geometric object in our proofs of
Theorem~\ref{thm:NN} and~\ref{thm:qNN}.
Theorem~\ref{thm:ER} is one of the few results on isometrically embedding a
geometric simplex into effective resistance, where the underlying
electrical network is not apparent. It also establishes links between
$q$-screw simplices and the considerably large field of spectral graph theory,
which we hope sets the stage for fertile future work.
Theorem~\ref{thm:l1} generalizes the $q$-screw simplex embeddability result
of Schoenbreg and Von Neumann, and mirrors a famous theorem of Schoenberg.
To the author's knowledge, this theorem was not known or even guessed
before.
