
\section{Fractional Laplacian}

In this section, we prove that the $q$-screw simplex distances arise as
effective resistance of the Fractional Laplacian, for powers $s = -1/2-1/q  $,
when $q > 2$. 

Preliminaries: Fractional Laplacian. 

We present two definitions: the first definition is based on taking the limit
of graph Laplacians raised to the fractional power (which are known in folklore
to be graph Laplacians themselves), and the second definition is based on
taking fractional powers of the Laplacian differential operator when the latter
is written in terms of its Eigenvectors, the Fourier bases.

In this work, we build on Von Neumann and Schoenberg’s proof of embeddability
of the $q$-screw simplex. Their work (slightly simplified by the authors of
this paper) shows that the $q$-screw simplex for $q > 1$ can be embedded in
infinite dimensional Hilbert space, by the embedding $f: mathbb{R} \rightarrow
L_2$ defined as:

\begin{align}\label{eq:screw-embedding} f(x) = \frac{e^{i\omega
x}}{\omega^{1/2+1/q}} \end{align} Where $f(x)$ is a function in the variable
$\omega$.

The proof of their embedding hinges on the following remarkable integral
formula: \[ \| x_1 – x_2 \|_2^{1/q} = \frac{sin^2(\omega(x_1-x_2)
)}{\omega^{1+2/q}},  \] the left hand side of which is the norm for
Equation~\ref{eq:screw-embedding}. This integral formula is a classical
integral formula~\cite{}, and can be proven using Jordan’s integration theorem
from complex analysis~\cite{,}.

Notice that the step function $S_x$, which is $1$ between $-x$ and $x$ and $0$
elsewhere, can be written in Fourier bases as: \begin{align} \label{eq:step}
S(x) = frac{e^{i\omega x}}{\omega}.  \end{align} Equation~\ref{eq:step} is a
classic result, dating back to the earliest days of functional analysis.
However, this formulation compared with Equation~\ref{eq:screw-embedding}
practically invites us to use the Fractional Laplacian, when viewed through the
Eigenvector lens in Section~\ref{sec:frac-lap-eigenvector}.

Therefore, we can write Equation~\ref{eq:screw-embedding} as: \[ f(x) =
\Delta^{1/4-1/(2q))} Step = \Delta^{1/4-1/(2q)} (\Delta^{-1/2} \delta_{-x} -
\delta_{x}) = \Delta^{-1/4-1/(2q)} (\delta_{-x} - \delta_{x}) \] In the above
expression, $\delta_x$ represents the Dirac Delta function. DEFINE $\Delta$ in
this case!!!! Here, the second part of the equation is a standard manipulation
in differential equations, as $\Delta^{1/2}$ is conceptually similar to the
integral operator. For more on manipulations with this fractional Laplacian
operator, see~\ref{}.

And thus $|x-y|^{2/q} = ||f(x)||_2^2$ can be written as: \begin{align}
(\delta_{-x} - \delta_{x})^T \cdot \Delta^{-1/2-1/q} \cdot (-\delta_{-x} –
\delta_{x}) \end{align} Which is an effective resistance distance. This can be
seen since $\Delta^{-1/2-1/q}$ can be written as the limit of fractional graph
Laplacians (which are in turn graph Laplacians, by Lemma~\ref{}). Given a
finite screw simplex, our distance is thus the limit of the Schur complement of
these graph Laplacians onto a finite point set, which is the limit of a
sequence of graph Laplacians. It can be seen easily that such graph Laplacians
must converge, and the limit of this convergence is a graph Laplacian whose
effective resistance distance are the screw simplex distances. This proves
Theorem~\ref{thm:embed}. 
