\section{Exactly Computing Nearest Neighbor Metrics, and $q$-NN metrics on
small point sets}\label{sec:metric-equality}

In this section, we prove our main theorems, Theorem~\ref{} and~\ref{},
wihch shows how to exactly compute Nearest Neighbor metrics and $q$-NN
metrics for small point sets.
We prove that the edge-squared metric exactly equals the nearest neighbor
metric on any point set, and that the $q$-edge power metric equals the $q$-NN
metric for any set of four points or less. We also conjecture that the $q$-edge
power metric always equals the $q$-NN metric. Even though the number of points
we handle for the general $q$-edge power metric is quite small, this still
represents progress; exactly computing the NN or $q$-NN metric even for four
points in two dimensions requires dealing with uncountably number of paths
through space. Furthermore, we provide a discrete criterion for testing whether
the $q$-NN metric equals the $q$-edge power metric, which may be useful for
future work.

Our tools for proving both theorems will be use of min-cost flows generated
from a conservative vector field. This is the core idea that lets us surmount
the difficulties in dealing with uncountably many paths. We hope that our ideas
may be more generally applicable to various metrics.  

Notice that the Nearest Neighbor cost is upper bounded by the edge-squared
metric. This can be done by purely considering Nearest Neighbor paths that
are piecewise linear and go straight from data point to data point. The
analagous result holds for $q$-Nearest Neighbor metrics. Therefore, we only
need to prove that the Nearest Neighbor cost is lower bounded by the
edge-squared metric.

To do this, we build a graph $G'$ from our point set, which can
conceptually be thought of as the edge-squared graph with additional
Steiner points. We will show using conservative vector fields and flows
that the Nearest Neighbor cost of any path from $a$ to $b$ is bounded below
by the shortest path from $a$ to $b$ in $G'$. If the shortest path in $G'$
were always equal to the shortest path in the edge-squraed graph $G$, then
we'd be done.

However, this is not the case in general. However, it does turn out to be
the case always on a $2$-screw simplex. Remarkably, we show that proving
this equality on the $2$ screw simplex is sufficient to prove it for any
point set. We show this via the Lipschitz extension theorem and a simple
BFS. This proves Theorem~\ref{thm:NN}

All our techniques generalize to $q$-NN metrics (when being related to $q$-edge power
metrics), except the equality
between shortest paths on $G'$ and $G$ is less clear for $q$-screw
simplices. We prove that this equality holds when there are four points in
three dimension, and conjecture (with computational evidence, but no proof) that this
equality holds for any point set. Thus, we have proven
Theorem~\ref{thm:qNN}, and provided a discrete criterion that would imply
Conjecture~\ref{conj:qNN} holds.

The gap in Theorem~\ref{thm:NN} and~\ref{thm:qNN} is due to the simple
geometric structure of the $2$-screw simplex, which is known to have a
beautiful embedding into the corners of a rectangular box.

\subsection{Detailed Proof Strategy}

Given points $x_0, x_1, \ldots x_n$, our strategy is to build a graph $G'$ with
vertices $v_0, v_1, \ldots v_{n-1}$, and $v_S$ for any subset $S$ of
$\{0,1,2,3,4 \ldots n-1 \}$. Here, $v_S$ is a vertex representing the
circumcenter of the vertex set $\{x_s: s \in S\}$. Graph $G'$ then has some
edges (connected in a fashion to be detailed later), and costs on these edges.
Graph $G'$ is similar to the barycentric subdivision on a graph, used
in~\cite{}.

Tto lower bound any nearest neighbor path in our original point set, we will
lower bound it by some unit flow (DFEINE) from $v_0$ to $v_n$ in $G'$. However,
note that the unit flow from $v_0$ to $v_n$ is lower bounded by the shortest
path (where lengths of edges are the same as their costs) from $v_0$ to $v_n$
in $G'$. If the shortest path in $G'$ is equal to the $q$-edge power cost of
going from $x_0$ to $x_n$, then we have completed our proof.

To lower bound the cost of the $q$-NN metric with a unit flow, we aim to build
a potential function $\Conv:\mathbb{R}^n \rightarrow \mathbb{R}^{2^n}$ such
that $\Conv(p)$ is a vector whose entries sum to one. Here, $\Conv(p)$
represents a convex combination of the vertices of $G’$. We construct $\Conv$
such that $\Conv(v_S) = e_S$, where $e_S$ represents the unit vector with $1$
in the coordinate indexed by $S$, and $0$ elsewhere. Additionally, $\Conv$ has
the property that the $q$-NN cost of any path from $x \in \mathbb{R}^n$ to
$x+\Delta(x) \in \mathbb{R}^n$ (with respect to point set $x_0, \ldots x_n$) is
bounded below by the min cost flow on $G’$ satisfying demands
$\Conv(x+\Delta(x)) - \Conv(x)$. If we can construct such a function $\Conv$,
then the $q$-NN path from $v_0$ to $v_n$ is thus lower bounded by the min cost
unit flow from $v_0$ to $v_n$. Therefore, we have successfully lower bounded
the $q$-NN cost of a path from $x_0$ to $x_n$ with the shortest path in $G’$.

Therefore, the remainder of the section is devoted to two tasks: first,
constructing a $G’$ such that the shortest path in $G’$ is equal to the
$q$-edge power distance from $x_0$ to $x_n$. Second, constructing a function
$\Conv$ satisfying the properties listed above.

When $q = 2$, we can do this for any $n$, where $x_0, \ldots x_n$ form a
$2$-screw simplex. This will prove Theorem~\ref{}. When $q > 2$, we can do this
for any four points that are vertices of a $q$-screw simplex. This will prove
Theorem~\ref{}. We further conjecture our approach will work for any $q > 2$
and $n$.


\subsection{Construction of $G’$}

Let $G'$ be a graph on $\mathbb{R}^{2^n}$, with vertices $v_{S}$ for all $S
\subset \{0,1,\ldots n-1\}$. Connect vertices $v_S$ and $v_T$ for $S, T \subset
\{0, 1, \ldots n-1\}$ iff sets $S$ and $T$ differ by exactly one element. We
assign this edge a cost of $|R_{S}^q - R_{T}^q|$, where $R_{S}$ is the
circumradius of points $\{x_s : s \in S\}$.


\begin{lemma}\label{lem:flow} There exists a function $\Conv$ such that for
any
path piece $\phi$ running from $x$ and $x+\Delta(x)$, $\len_{qN}(\phi) \geq
MCF_{G'}(B'(x+\Delta(x))-B'(x))$. Here, $MCF_{G'}(d)$ for $d \in
\mathbb{R}^{2^n}$ represents the minimum cost flow on $G'$ satisfying demands
$d$ on the vertices of $G'$.

\end{lemma}

\begin{lemma}\label{lem:radius} Let $S = s_1 < s_2 < \ldots s_{|S|}$, and $T =
t_1 < \ldots \leq t_{|T|}$, where $S, T \subset \{0,1,\ldots n-1\}$ and $S$ and
$T$ differ by exactly one element. Let $R_S$ denote the circumradius of the
$q$-screw simplex formed by $s_1, s_2, \ldots s_n$. ~\tim{Define what
'being formed by' means.} If

\begin{align}\label{eq:radius} |(R_{S})^q-(R_{T})^q| \geq
|(s_{|S|}-s_1)/2^q-(t_{|T|}-t_1)/2^q|, \end{align}

then the shortest path in $G'$ equals the shortest path in $G$.

\end{lemma}

Note that this inequality is always an exact equality for $q=2$. The proof of
this lemma is combinatorial:

\begin{proof}

\end{proof}

Therefore, if Equation~\ref{eq:radius} holds for any set of $k$ points, and
Lemma~\ref{lem:flow} and~\ref{lem:radius} are proven, then we've proven
Theorem~\ref{thm:general} for any $k$ element point set.

\textbf{Proof of Lemma~\ref{lem:flow}}

\textbf{Proof of Lemma~\ref{lem:radius}}

\begin{lemma} $SP_{G'}(v_0, v_n) \leq SP_G(x_0, x_n)$

\end{lemma}

\begin{proof}

The distance from $v_0$ to $v_{\{0,n-1\}}$ on the graph is equal to
$u_{n-1}-u_0$ (NORMALIZE THINGS PROPERLY, INDEX THINGS PROPER, SET COMMON
VARIABLE NAMES), and the distance from $v_{\{0, n-1\}}$ to $v_{n-1}$ is the
same. Thus, this path from $v_0$ to $v_n$ equals $SP_G(x_0, x_n)$, as the
latter is trivially equal to $u_{n-1} - u_0$.

\end{proof}

Thus, it suffices to prove that

\[ SP_{G'}(v_0, v_n) \geq SP_G(x_0, x_n) \]

%%%% GAP

\subsection{Construction of $\Conv$}

In this section, we prove we Lemma~\ref{lem:flow}. We construct a function
$\Conv:\mathbb{R}^n \rightarrow \mathbb{R}^{2^n}$ assigning every point in
Euclidean space to a vector, representing a convex combination of vertices
in $G'$. We will build $\Conv$ separately for each Voronoi cell, and show
it is piecewise continuous across boundaries of Voronoi cells. 
However, to simplify our arguments, we further divide up each Voronoi cell
into simplices similar to the dissection in a barycentric subdivision.
However, rather than barycenters, we use circumcenters, so we are
more-precisely dividing up the $q$-screw simplex with a circumcentric
subdivision. \tim{define circumcentric/barycentric subdivision}

Let $p_{S}$ be the circumcenter of points $\{p_s | s \in S\}$ when $S
\subset \{1, 2, \ldots n\}$. A simplicial cell in the circumcentric subdivision is
defined by a permutation $a_0, a_1, \ldots a_{n-1}$ of $\{0, 1, \ldots
n-1\}$ as follows: let
$A_i = \{a_0, a_1, \ldots a_{i}\}$. Then the vertices $\{p_{A_i} | 0 \leq i
< n\}$ are the vertices of the simplicial cell.  \tim{Move this definition
upwards somewhere?} Each Voronoi cell is the disjoint~\tim{not exactly
disjoint} union of some of
these cells, and the cells partition the simplex \tim{This isn't quite
true: its only true for special geometry cells, but we don't particularly
care... how can I make this point clear?}. Our general strategy is to
create $\Conv$ on each simplicial cell, and show that is piecewise
continuous across the boundaries of simplicial cells.

To simplify our notation, we define $\pr_i$ to be $p_{A_i}$. \tim{Make sure
all points are $p$, not $x$. $\xr_k$ is later used to coordinate-wise split
up $x$}.

%Define $\pr, \xr, \Rr, \Convr$ here. Points $\pr$ numbered from $0$ to $k-1$.
%Here, $\xr$ is numbered $1$ through $k-1$. $B$ is $0$ through $k-1$
%indexed.


By the construction of circumcentric subdivisions, 
the line $\pr_{i}\pr_{i+1}$ is perpendicular to
$\pr_{i+1}\pr_{i+2}$ for all $i$. These lines define a natural
orthonormal coordinate axis. Thus,
for any $\xr$ in the convex hull of $\pr_i$, we can write $\xr$ in
coordinates $(\xr_0, \xr_1, \ldots \xr_{n-2})$, where the $i^{th}$ coordinate
axis is parallel to $\pr_i \pr_{i+1}$.

Next, we introduce how $\Convr$ is defined on the vertices $\pr_{i}$. Here,
$\Convr_i$ is shorthand for the value of $\Convr$ on $\pr_{i}$.

\begin{align}
\Convr(\xr)_i = \frac{
\left(\sum_{s=0}^{i-1} {\xr_s}^2\right)^{q/2} -\left(\sum_{s=0}^{i-2}
{\xr_s}^2\right)^{q/2}
}
{
\Rr_i^q - \Rr_{i-1}^q
} - \sum_{i < j < n} \Convr(\xr)_j
\end{align}

for all $1 \leq i \leq n-1$, and \tim{make sure you replace all k's with
n's, and claim that you're dealing with simplices of full dimension. You
may want to say that the proof is counterintuitive since simplices of full
dimension are usually considered harder?}

\[ \Convr(\xr)_0 = 1 - \sum_{0 < j < n} \Convr(\xr)_j
\]

The key feature about $\Convr$ is that

\[\sum_{j=i}^n \Convr(\xr)_j= \frac{
\left(\sum_{s=0}^{i-1} {\xr_s}^2\right)^{q/2} -\left(\sum_{s=0}^{i-2}
{\xr_s}^2\right)^{q/2}
}
{
\Rr_i^q - \Rr_{i-1}^q
}\]

for all $i > 0$, and for $i = 0$ the LHS evaluates to $1$.

Since we defined this function piecewise, we need to check that this
function is piecewise continuous.

\begin{lemma} If $\xr$ is on a face of $\pr_0, \ldots \pr_n$, then
$\Convr(\xr)$ has non-zero coordinates only on that face. Furthermore, the
coordinates depend only on SOMETHING.

\end{lemma}

\begin{proof}

\end{proof}

Now we are ready to prove our core lemma:

\begin{lemma} For $x$ and $x+\Delta(x)$ in the convex hull of $\pr_1,
\ldots \pr_k$, the distance:

\[ \|x-\pr_1\|^{q-1} \cdot \|\Delta(x)\| \geq
F\left(\Convr(x+\Delta(x))-\Convr(x)\right)\]

where $F(d)$ is the unique cost of a flow satisfying demand $d \in
\mathbb{R}^n$ on $\Gr$.

\end{lemma}

Here, the left hand side represents the $q$-NN cost of a path piece from
$x$ to $x+\Delta(x)$, and the right hand side is the unique cost of the
induced flow on graph $G'$, with the restriction that the flow is only
nonzero on the vertices $\pr_i\pr_{i+1}$ for any $0 \leq i < n$. This flow
is unique since we forced our flow to be non-zero only on the edges
$\pr_i\pr_{i+1}$, which form a line graph; and for any set of demands on
vertices of a line, there is a unique flow satisfying those demands.

\begin{proof} For any edge $\pr_i\pr_{i+1}$, the cost of a flow (satisfying
some set of demands whose sum is $0$) on that edge is the absolute value of
the sum of the demands on vertices $\pr_{i+1} \pr_{i+2}, \ldots \pr_n$,
multiplied by the cost of the edge from $\pr_i$ to $\pr_{i+1}$. This
quantity comes out to be:

\begin{align} &\left(\Rr_{i+1}^q-\Rr_{i}^q \right) \sum_{j=i+1}^n \Convr(\xr)_j
\\
\label{eq:flow-cost}
&=\left(\sum_{s=0}^{i-1} {\xr_s}^2\right)^{q/2} -\left(\sum_{s=0}^{i-2}
{\xr_s}^2\right)^{q/2}.
\end{align}

As $\Delta(x)$ goes to $0$, the change in Expression~\ref{eq:flow-cost} is

\begin{align}
\begin{split}
\label{eq:flow-cost-on-edge}
&
\left(q\xr_0 \left(\sum_{s=0}^{i-1} \xr_s^2 \right)^{q/2-1} -
q\xr_0\left(\sum_{s=0}^{i-2} \xr_s^2 \right)^{q/2-1} \right) \Delta(\xr)_0
\\
&+
\left(q\xr_1 \left(\sum_{s=0}^{i-1} \xr_s^2 \right)^{q/2-1} -
q\xr_1\left(\sum_{s=0}^{i-2}\xr_s^2 \right)^{q/2-1} \right) \Delta(\xr)_1
\\
&+ \ldots
\\
&+
\left(q\xr_{i-2} \left(\sum_{s=0}^{i-1} \xr_s^2 \right)^{q/2-1} -
q\xr_{i-2}\left(\sum_{s=0}^{i-2}\xr_s^2 \right)^{q/2-1}
\right)\Delta(\xr)_{i-2}
\\
&+
\left(q\xr_{i-1} \left(\sum_{s=0}^{i-1} \xr_s^2
\right)^{q/2-1}\right)\Delta(\xr)_{i-1},
\end{split}
\end{align}
Since 
\[
q\xr_j\left(\sum_{s=0}^{i-1} \xr_s^2\right)^{q/2-1} - q\xr_j\left(\sum_{s=0}^{i-2}
\xr_s^2 \right)^{q/2-1}\]
is always non-negative (only when $q \leq 2$), we get that the absolute value of
Expression~\ref{eq:flow-cost-on-edge} is bounded above by:
\begin{align}
\begin{split}
\label{eq:flow-cost-upper-bound}
&
\left(q\xr_0 \left(\sum_{s=0}^{i-1} \xr_s^2 \right)^{q/2-1} -
q\xr_0\left(\sum_{s=0}^{i-2} \xr_s^2 \right)^{q/2-1} \right)
|\Delta(\xr)_0|
\\
&+
\left(q\xr_1 \left(\sum_{s=0}^{i-1} \xr_s^2 \right)^{q/2-1} -
q\xr_1\left(\sum_{s=0}^{i-2}\xr_s^2 \right)^{q/2-1} \right) |\Delta(\xr)_1|
\\
&+ \ldots
\\
&+
\left(q\xr_{i-2} \left(\sum_{s=0}^{i-1} \xr_s^2 \right)^{q/2-1} -
q\xr_{i-2}\left(\sum_{s=0}^{i-2}\xr_s^2 \right)^{q/2-1}
\right)|\Delta(\xr)_{i-2}|
\\
&+
\left(q\xr_{i-1} \left(\sum_{s=0}^{i-1} \xr_s^2
\right)^{q/2-1}\right)|\Delta(\xr)_{i-1}|,
\end{split}
\end{align}

Expression~\ref{eq:flow-cost-upper-bound} is an upper bound on the cost of
a flow along edge $\vr_i\vr_{i+1}$\tim{Is this the right indexing for
edges?} induced by a path from $x$ to $x+\Delta(x)$. Now we sum this across all $i$ to get an overall cost upper bound, and group by
$\Delta(\xr)_i$ for fixed i. The sum telescopes beautifully, and we get:
\begin{align}
&\left(q\xr_0\left(\sum_{s=0}^{n-2} \xr_s^2\right)^{q/2-1}\right)| \Delta(\xr)_0|
\\&+
\left(q\xr_1\left(\sum_{s=0}^{n-2} \xr_s^2\right)^{q/2-1}\right)| \Delta(\xr)_1|
\ldots
\\& +
\left(q\xr_{n-1}\left(\sum_{s=0}^{n-2} \xr_s^2\right)^{q/2-1}\right)| \Delta(\xr)_{n-2}|
\end{align}
This expression, by Cauchy Schwarz, is upper bounded by
\[
\sqrt{\sum_{s=0}^{n-2} \Delta(\xr)_s^2} \cdot \left(q\sqrt{\sum_{s=0}^{n-2}
\xr_s^{q-1}}\right)
\]
Which is exactly the $q$-NN distance.

\end{proof}

Note that this function is piecewise continuous on the boundary. Therefore,
we have shown that the $q$-NN cost of any path piece is less than the
min-cost flow on $G’$ satisfying
$\Conv\left(x+\Delta(x)\right)-\Conv\left(x\right)$ for infinitesimal
$\Delta(x)$, as desired.

So far, the only property our flow construction used is that the points
$x_0, x_1, \ldots x_n$ have Voronoi subdivisions defined by $\pr_{a_0}$,
$\pr_{a_0a_1}$, \ldots $\pr_{a_0a_1\ldots a_k}$, for some $a_0, ldots a_k
\subset \{0, 1, \ldots n\}$. (DOES THIS WORK FOR ANY GEOMETRY, OR DO I NEED
THE INTERIOR CIRCUMCENTER PROPERTY?).

Thus, we have proven a core lemma:

\begin{lemma}\label{lem:qNN-GPrime}

The $q$-NN distance between two points in a point set is lower bounded by
the shortest path between the two corresponding points in $G’$. Here, $G’$
is constructed as in Definition~\ref{def:GPrime}

\end{lemma}

We now prove the following two lemmas, completing our proof of
Theorem~\ref{thm:NN} and Theorem~\ref{thm:qNN} respectively.

\begin{lemma}\label{lem:edge-squared-GPrime} Let $G$ be the edge-squared graph
(DEFINE), and let $G’$ be defined as in Definition~\ref{def:GPrime} for $q=2$.
The shortest path in $G’$ is the same as the shortest path in $G$, when the
initial point set generating $G$ and $G’$ is a $2$-screw simplex.

\end{lemma}

\begin{lemma}\label{lem:q-edge-power-GPrime} Let $q > 2$. Let $G$ be the
$q$-edge power graph, and $G’$ be defined as in Definition~\ref{def:GPrime}
(MAKE SURE THE DEFINITION IS Q DEPENDENT). The shortest path in $G’$ is the
same as the shortest path in $G$, when the initial point set generating $G$
and $G’$ is a $q$-screw simplex with $4$ points.

\end{lemma}

Combined with Theorem~\ref{thm:screw-simplex-reduction},
Lemmas~\ref{edge-squared-GPrime} and~\ref{q-edge-power-GPrime} prove
Theorems~\ref{thm:NN} and~\ref{thm:qNN} respectively. Moreover, we make the
following conjecture, which we have some computational evidence for (See
Appendix~\ref{} for details):

\begin{conjecture}\label{conj:qNN}

For $q>2$, let $G$ and $G’$ be defined as in
Lemma~\ref{lem:q-edge-power-GPrime}. Then the shortest path in $G’$ is the same
as the shortest path in $G$.

\end{conjecture}

If this were true, it would prove that the $q$-edge power metric and the
$q$-NN metric were equal for all $q>2$.


Now, our proof will proceed in two parts. First, we construct a function
$B':\mathbb{R}^n \rightarrow \mathbb{R}^{2^n}$ assigning every point in
Euclidean space to a vector, representing a convex combination of vertices in
$G'$. We build $B'$ such that $B'(v_S)= e_S$, where $e_S$ is the unit vector
with $1$ in the dimension indexed by $S$. Next, we lower bound the
Nearest-Neighbor cost of an infinitesimal path piece. If the infinitesimal path
piece runs from $x \in \mathbb{R}^n$ to $x+\Delta(x)$, where both $x$ and
$x+\Delta(x)$ are in the same circumcentric subdivision (DEFINE), then we lower
bound it with some flow on the graph $G'$ satisfying demands $B'(x+\Delta(x)) -
B'(x)$.

If we can find a cost function $B'$ with these properties, we can integrate
over all the infinitesimal path pieces to show that the $p$-NN cost of a path
from $x_0$ to $x_{n-1}$ is lower bounded by some flow on $G'$. This is a unit
flow from $x_0$ to $x_n$. If we furthermore have that the shortest path on $G'$
equals the shortest path on $G$, then we have:

$$NN(path) \geq Q' \geq SP_{G'}(v_0, v_n) = SP_{G}(x_0, x_n).$$

Here, $Q'$ represents the cost of some flow from $v_0$ to $v_n$ on $G'$. Here,
$SP_{G'}(v_0, v_n)$ is the shortest path from $v_0$ to $v_n$ on $G'$ and
$SP_G(x_0, x_n)$ is the shortest path from $x_0$ to $x_n$ on the q-edge power
graph of the $q$-screw simplex.

Our proof then consists of two parts: the first is finding a graph $G'$ and
showing $SP_{G'}(v_0, v_n) = SP_G(x_0, x_n)$ (STANDARDIZE INDICES). The second
is finding a function $B'$ satisfying our desired properties listed in our
strategy. In the remainder of this section, we build our graph $G'$ and
establish

a discrete, sufficient-but-not-necessary criterion for when $SP_{G'}(v_0, v_n)
= SP_G(x_0, x_n)$. Then we build a function $B'$ and prove that $NN(path) \geq
Q'$, where the path runs from $v_0$ to $v_1$.

We then show that this necessary-but-not-sufficient criterion holds for any $5$
point $q$-screw simplex, thereby showing that the $q$-edge power metric equals
the $q$-NN metric for all $5$ point sets. We further conjecture that this
criterion holds for any $n$ points, but the authors are currently unable to
prove it.

Therefore, we have provided a sufficient but not necessary criterion for which
the $q$-edge power metric equals the $q$-NN Metric on the $q$-screw simplex.
Furthermore

Then we establish a discrete criterion in which $G' \geq G$. We then prove this
criterion holds for all sets of $5$ points in arbitrary dimension, and
conjecture that it holds for all sets of $n$ points in arbitrary dimension.

\subsection{Construction of $\Conv$}

In this section, we prove we Lemma~\ref{lem:flow}. We construct a function
$\Conv:\mathbb{R}^n \rightarrow \mathbb{R}^{2^n}$ assigning every point in
Euclidean space to a vector, representing a convex combination of vertices
in $G'$. We will build $\Conv$ separately for each Voronoi cell, and show
it is piecewise continuous across boundaries of Voronoi cells. 
However, to simplify our arguments, we further divide up each Voronoi cell
into simplices similar to the dissection in a barycentric subdivision.
However, rather than barycenters, we use circumcenters, so we are
more-precisely dividing up the $q$-screw simplex with a circumcentric
subdivision. \tim{define circumcentric/barycentric subdivision}

Let $p_{S}$ be the circumcenter of points $\{p_s | s \in S\}$ when $S
\subset \{1, 2, \ldots n\}$. A simplicial cell in the circumcentric subdivision is
defined by a permutation $a_0, a_1, \ldots a_{n-1}$ of $\{0, 1, \ldots
n-1\}$ as follows: let
$A_i = \{a_0, a_1, \ldots a_{i}\}$. Then the vertices $\{p_{A_i} | 0 \leq i
< n\}$ are the vertices of the simplicial cell.  \tim{Move this definition
upwards somewhere?} Each Voronoi cell is the disjoint~\tim{not exactly
disjoint} union of some of
these cells, and the cells partition the simplex \tim{This isn't quite
true: its only true for special geometry cells, but we don't particularly
care... how can I make this point clear?}. Our general strategy is to
create $\Conv$ on each simplicial cell, and show that is piecewise
continuous across the boundaries of simplicial cells.

To simplify our notation, we define $\pr_i$ to be $p_{A_i}$. \tim{Make sure
all points are $p$, not $x$. $\xr_k$ is later used to coordinate-wise split
up $x$}.

%Define $\pr, \xr, \Rr, \Convr$ here. Points $\pr$ numbered from $0$ to $k-1$.
%Here, $\xr$ is numbered $1$ through $k-1$. $B$ is $0$ through $k-1$
%indexed.


By the construction of circumcentric subdivisions, 
the line $\pr_{i}\pr_{i+1}$ is perpendicular to
$\pr_{i+1}\pr_{i+2}$ for all $i$. These lines define a natural
orthonormal coordinate axis. Thus,
for any $\xr$ in the convex hull of $\pr_i$, we can write $\xr$ in
coordinates $(\xr_0, \xr_1, \ldots \xr_{n-2})$, where the $i^{th}$ coordinate
axis is parallel to $\pr_i \pr_{i+1}$.

Next, we introduce how $\Convr$ is defined on the vertices $\pr_{i}$. Here,
$\Convr_i$ is shorthand for the value of $\Convr$ on $\pr_{i}$.

\begin{align}
\Convr(\xr)_i = \frac{
\left(\sum_{s=0}^{i-1} {\xr_s}^2\right)^{q/2} -\left(\sum_{s=0}^{i-2}
{\xr_s}^2\right)^{q/2}
}
{
\Rr_i^q - \Rr_{i-1}^q
} - \sum_{i < j < n} \Convr(\xr)_j
\end{align}

for all $1 \leq i \leq n-1$, and \tim{make sure you replace all k's with
n's, and claim that you're dealing with simplices of full dimension. You
may want to say that the proof is counterintuitive since simplices of full
dimension are usually considered harder?}

\[ \Convr(\xr)_0 = 1 - \sum_{0 < j < n} \Convr(\xr)_j
\]

The key feature about $\Convr$ is that

\[\sum_{j=i}^n \Convr(\xr)_j= \frac{
\left(\sum_{s=0}^{i-1} {\xr_s}^2\right)^{q/2} -\left(\sum_{s=0}^{i-2}
{\xr_s}^2\right)^{q/2}
}
{
\Rr_i^q - \Rr_{i-1}^q
}\]

for all $i > 0$, and for $i = 0$ the LHS evaluates to $1$.

Since we defined this function piecewise, we need to check that this
function is piecewise continuous.

\begin{lemma} If $\xr$ is on a face of $\pr_0, \ldots \pr_n$, then
$\Convr(\xr)$ has non-zero coordinates only on that face. Furthermore, the
coordinates depend only on SOMETHING.

\end{lemma}

\begin{proof}

\end{proof}

Now we are ready to prove our core lemma:

\begin{lemma} For $x$ and $x+\Delta(x)$ in the convex hull of $\pr_1,
\ldots \pr_k$, the distance:

\[ \|x-\pr_1\|^{q-1} \cdot \|\Delta(x)\| \geq
F\left(\Convr(x+\Delta(x))-\Convr(x)\right)\]

where $F(d)$ is the unique cost of a flow satisfying demand $d \in
\mathbb{R}^n$ on $\Gr$.

\end{lemma}

Here, the left hand side represents the $q$-NN cost of a path piece from
$x$ to $x+\Delta(x)$, and the right hand side is the unique cost of the
induced flow on graph $G'$, with the restriction that the flow is only
nonzero on the vertices $\pr_i\pr_{i+1}$ for any $0 \leq i < n$. This flow
is unique since we forced our flow to be non-zero only on the edges
$\pr_i\pr_{i+1}$, which form a line graph; and for any set of demands on
vertices of a line, there is a unique flow satisfying those demands.

\begin{proof} For any edge $\pr_i\pr_{i+1}$, the cost of a flow (satisfying
some set of demands whose sum is $0$) on that edge is the absolute value of
the sum of the demands on vertices $\pr_{i+1} \pr_{i+2}, \ldots \pr_n$,
multiplied by the cost of the edge from $\pr_i$ to $\pr_{i+1}$. This
quantity comes out to be:

\begin{align} &\left(\Rr_{i+1}^q-\Rr_{i}^q \right) \sum_{j=i+1}^n \Convr(\xr)_j
\\
\label{eq:flow-cost}
&=\left(\sum_{s=0}^{i-1} {\xr_s}^2\right)^{q/2} -\left(\sum_{s=0}^{i-2}
{\xr_s}^2\right)^{q/2}.
\end{align}

As $\Delta(x)$ goes to $0$, the change in Expression~\ref{eq:flow-cost} is

\begin{align}
\begin{split}
\label{eq:flow-cost-on-edge}
&
\left(q\xr_0 \left(\sum_{s=0}^{i-1} \xr_s^2 \right)^{q/2-1} -
q\xr_0\left(\sum_{s=0}^{i-2} \xr_s^2 \right)^{q/2-1} \right) \Delta(\xr)_0
\\
&+
\left(q\xr_1 \left(\sum_{s=0}^{i-1} \xr_s^2 \right)^{q/2-1} -
q\xr_1\left(\sum_{s=0}^{i-2}\xr_s^2 \right)^{q/2-1} \right) \Delta(\xr)_1
\\
&+ \ldots
\\
&+
\left(q\xr_{i-2} \left(\sum_{s=0}^{i-1} \xr_s^2 \right)^{q/2-1} -
q\xr_{i-2}\left(\sum_{s=0}^{i-2}\xr_s^2 \right)^{q/2-1}
\right)\Delta(\xr)_{i-2}
\\
&+
\left(q\xr_{i-1} \left(\sum_{s=0}^{i-1} \xr_s^2
\right)^{q/2-1}\right)\Delta(\xr)_{i-1},
\end{split}
\end{align}
Since 
\[
q\xr_j\left(\sum_{s=0}^{i-1} \xr_s^2\right)^{q/2-1} - q\xr_j\left(\sum_{s=0}^{i-2}
\xr_s^2 \right)^{q/2-1}\]
is always non-negative (only when $q \leq 2$), we get that the absolute value of
Expression~\ref{eq:flow-cost-on-edge} is bounded above by:
\begin{align}
\begin{split}
\label{eq:flow-cost-upper-bound}
&
\left(q\xr_0 \left(\sum_{s=0}^{i-1} \xr_s^2 \right)^{q/2-1} -
q\xr_0\left(\sum_{s=0}^{i-2} \xr_s^2 \right)^{q/2-1} \right)
|\Delta(\xr)_0|
\\
&+
\left(q\xr_1 \left(\sum_{s=0}^{i-1} \xr_s^2 \right)^{q/2-1} -
q\xr_1\left(\sum_{s=0}^{i-2}\xr_s^2 \right)^{q/2-1} \right) |\Delta(\xr)_1|
\\
&+ \ldots
\\
&+
\left(q\xr_{i-2} \left(\sum_{s=0}^{i-1} \xr_s^2 \right)^{q/2-1} -
q\xr_{i-2}\left(\sum_{s=0}^{i-2}\xr_s^2 \right)^{q/2-1}
\right)|\Delta(\xr)_{i-2}|
\\
&+
\left(q\xr_{i-1} \left(\sum_{s=0}^{i-1} \xr_s^2
\right)^{q/2-1}\right)|\Delta(\xr)_{i-1}|,
\end{split}
\end{align}

Expression~\ref{eq:flow-cost-upper-bound} is an upper bound on the cost of
a flow along edge $\vr_i\vr_{i+1}$\tim{Is this the right indexing for
edges?} induced by a path from $x$ to $x+\Delta(x)$. Now we sum this across all $i$ to get an overall cost upper bound, and group by
$\Delta(\xr)_i$ for fixed i. The sum telescopes beautifully, and we get:
\begin{align}
&\left(q\xr_0\left(\sum_{s=0}^{n-2} \xr_s^2\right)^{q/2-1}\right)| \Delta(\xr)_0|
\\&+
\left(q\xr_1\left(\sum_{s=0}^{n-2} \xr_s^2\right)^{q/2-1}\right)| \Delta(\xr)_1|
\ldots
\\& +
\left(q\xr_{n-1}\left(\sum_{s=0}^{n-2} \xr_s^2\right)^{q/2-1}\right)| \Delta(\xr)_{n-2}|
\end{align}
This expression, by Cauchy Schwarz, is upper bounded by
\[
\sqrt{\sum_{s=0}^{n-2} \Delta(\xr)_s^2} \cdot \left(q\sqrt{\sum_{s=0}^{n-2}
\xr_s^{q-1}}\right)
\]
Which is exactly the $q$-NN distance.

\end{proof}

Note that this function is piecewise continuous on the boundary. Therefore,
we have shown that the $q$-NN cost of any path piece is less than the
min-cost flow on $G’$ satisfying
$\Conv\left(x+\Delta(x)\right)-\Conv\left(x\right)$ for infinitesimal
$\Delta(x)$, as desired.

So far, the only property our flow construction used is that the points
$x_0, x_1, \ldots x_n$ have Voronoi subdivisions defined by $\pr_{a_0}$,
$\pr_{a_0a_1}$, \ldots $\pr_{a_0a_1\ldots a_k}$, for some $a_0, ldots a_k
\subset \{0, 1, \ldots n\}$. (DOES THIS WORK FOR ANY GEOMETRY, OR DO I NEED
THE INTERIOR CIRCUMCENTER PROPERTY?).

Thus, we have proven a core lemma:

\begin{lemma}\label{lem:qNN-GPrime}

The $q$-NN distance between two points in a point set is lower bounded by
the shortest path between the two corresponding points in $G’$. Here, $G’$
is constructed as in Definition~\ref{def:GPrime}

\end{lemma}

We now prove the following two lemmas, completing our proof of
Theorem~\ref{thm:NN} and Theorem~\ref{thm:qNN} respectively.

\begin{lemma}\label{lem:edge-squared-GPrime} Let $G$ be the edge-squared graph
(DEFINE), and let $G’$ be defined as in Definition~\ref{def:GPrime} for $q=2$.
The shortest path in $G’$ is the same as the shortest path in $G$, when the
initial point set generating $G$ and $G’$ is a $2$-screw simplex.

\end{lemma}

\begin{lemma}\label{lem:q-edge-power-GPrime} Let $q > 2$. Let $G$ be the
$q$-edge power graph, and $G’$ be defined as in Definition~\ref{def:GPrime}
(MAKE SURE THE DEFINITION IS Q DEPENDENT). The shortest path in $G’$ is the
same as the shortest path in $G$, when the initial point set generating $G$
and $G’$ is a $q$-screw simplex with $4$ points.

\end{lemma}

Combined with Theorem~\ref{thm:screw-simplex-reduction},
Lemmas~\ref{edge-squared-GPrime} and~\ref{q-edge-power-GPrime} prove
Theorems~\ref{thm:NN} and~\ref{thm:qNN} respectively. Moreover, we make the
following conjecture, which we have some computational evidence for (See
Appendix~\ref{} for details):

\begin{conjecture}\label{conj:qNN}

For $q>2$, let $G$ and $G’$ be defined as in
Lemma~\ref{lem:q-edge-power-GPrime}. Then the shortest path in $G’$ is the same
as the shortest path in $G$.

\end{conjecture}

If this were true, it would prove that the $q$-edge power metric and the
$q$-NN metric were equal for all $q>2$.


