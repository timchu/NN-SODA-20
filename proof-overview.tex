\subsection{Proof Overview}

Given points $x_0, x_1, \ldots x_n$, we build a graph $G'$ with
vertices $v_S$ for any subset $S$ of
$\{0,1,2,3,4 \ldots n-1 \}$. Here, $v_S$ is a vertex representing the
circumcenter of the vertex set $\{x_s: s \in S\}$. Graph $G'$ then has some
edges, connected in a fashion to be detailed later, and costs on these edges.
Graph $G'$ is related to the barycentric subdivision on a graph, used
in~\cite{}.

Tto lower bound the nearest neighbor cost of any path on our original point
set, we will lower bound it by some unit flow (DFEINE) from $v_0$ to $v_n$
in $G'$. However, note that the unit flow from $v_0$ to $v_n$ is lower
bounded by the shortest path (where lengths of edges are the same as their
costs) from $v_0$ to $v_n$ in $G'$. If the shortest path in $G'$ is equal
to the $q$-edge power cost of going from $x_0$ to $x_n$, then we have
completed our proof.

To lower bound the cost of the $q$-NN metric with a unit flow, we aim to build
a potential function $\Conv:\mathbb{R}^n \rightarrow \mathbb{R}^{2^n}$ such
that $\Conv(p)$ is a vector whose entries sum to one, supported on the
vertices of $G'$. Here, $\Conv(p)$
represents a convex combination of the vertices of $G'$. We construct $\Conv$
such that $\Conv(v_S) = e_S$, where $e_S$ represents the unit vector with $1$
in the coordinate indexed by $S$, and $0$ elsewhere. Additionally, we would
like $\Conv$ to have
the property that the $q$-NN cost of any path from $x \in \mathbb{R}^n$ to
$x+\Delta(x) \in \mathbb{R}^n$ for small $\Delta(x)$ is
bounded below by the min-cost flow on $G’$ satisfying demands
$\Conv(x+\Delta(x)) - \Conv(x)$. If we can construct such a function $\Conv$,
then the $q$-NN path from $v_0$ to $v_n$ is thus lower bounded by the min-cost
unit flow from $v_0$ to $v_n$, which is exactly the shortest path from
$v_0$ to $v_n$ in $G'$!
 Therefore, building $\Conv$ implies that we can lower bound the
the $q$-NN cost of a path from $x_0$ to $x_n$ with the shortest path in $G'$.

This bound holds for any initial point set $x_0, x_1, \ldots x_n$. However,
in general the shortest path on $G'$ and the shortest path on $G$ are not
the same. to overcome this, we show that it suffices to prove
Theorem~\ref{thm:NN} on $2$-screw simplices, and we show that on $2$-screw
simplices the shortest path on $G'$ is the shortest path on $G$.

Therefore, our proof is devoted to two things: showing the initial
inequality between $q$-NN metrics and shortest paths on $G'$, proving
that it suffices to show Theorem~\ref{thm:NN} on the $2$-screw simplex, and
proving that the shortest path between points on $G'$ and $G$ are the same
for the $2$ screw simplex. Our proof of Theorem~\ref{thm:qNN} follows the
same structure.


%show two things: first, that it suffices to prove our initial 
%Therefore, we are primarily devoted to two tasks: first,
%constructing a $G'$, and constructing a function
%$\Conv$ satisfying the properties listed above.  Later, we will see how to
%go from $G'$ to the $q$-edge power graph for all $n$ when $q=2$, and for $
%n\leq 4$ when $q > 2$. These combined will prove Theorems~\ref{thm:NN}
%and~\ref{thm:qNN}.
