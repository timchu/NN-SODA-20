\begin{abstract}
In the machine learning setting, distances between two datapoints in a
Euclidean point set are considered short if they are in the same data
cluster - even if their Euclidean distance is long. A simple metric with
this property is the Nearest neighbor metric. This metric is a
mainfold-based metric, and it and its
close variants have been studied in the past by multiple researchers.

One key problem on manifold metrics, dating back for four centuries, is
computing them exactly. The Nearest Neighbor metric
is defined as the infimum cost path over an
uncountable number of paths that can go 'anywhere' on a continuous
manifold. This makes exactly
computing the Nearest Neighbor metric challenging, even for a fixed
set of four points in two dimensions. In this paper, we overcome this
challenge by equating the Nearest Neighbor metric to a shortest-path
distance on a simple geometric graph, in all cases. We then
exactly compute a generalization of the Nearest Neighbor metric, called the
$q$-Nearest Neighbor metric, for small point sets.  Our tools include
conservative vector fields, Lipschitz extensions, minimum cost flows, and
barycentric subdivisions, all applied to a geometric object we call the
$q$-screw simplex. Our work considerably strengthens the work of Cohen et.
al., and shows the first non-trivial manifold metric that can be computed
exactly with discrete techniques. We can use our results to solve a range
of problems important to metrics, on the Nearest Neighbor metric: efficient
spanner computation, persistent homology, and behavior in the limit. We
further show links between this metric and notable problems like Euclidean
MST and single-linkage clustering.

The key geometric object in our proof, the $q$-screw simplex, was first
discovered by John Von Neumann and Issai Schoenberg. This simplex is
defined by taking $n$ points on a line, applying the $1/q$ power to all
pairwise distances, and isometrically embedding the resulting point set
into Euclidean space.  We prove results on these $q$-screw simplices that
are of independent interest: we show that they
have a deep connection to fractional Laplacians, a differential operator
that appears in a wide variety of physics and mathematical settings. We use
this connection to prove that the $q$-screw simplex embeds isometrically
into Effective resistance distance, and that its volume and circumcenter
can be approximated efficiently using a Laplacian determinant estimator and
Laplacian system solver respectively. 
%We also show that any finite $l_1$
%metric raised to the $1/q$ power is isometrically embeddable into $l_1$,
%mirroring a famous theorem by Schoenberg on $l_2$. 
\end{abstract}
