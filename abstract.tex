\begin{abstract}
A foundational hypothesis in machine learning is that datapoints can be
embedded into Euclidean space, and metrics can be generated on the
datapoints to solve a wide range of tasks. It is widely believed that these
metrics should have the
property that two points in the same dense cluster of datapoints should be
considered close, even if their Euclidean distance is far.
One simple metric with
this property is the Nearest neighbor metric. This metric is a
mainfold-based metric, and it and its
close variants have been studied in the past by multiple researchers.

One key problem on manifold metrics, dating back for four centuries, is
computing them exactly. The Nearest Neighbor metric
is defined as the infimum cost path over an
uncountable number of paths that can go 'anywhere' on a continuous
manifold. This makes exactly
computing the Nearest Neighbor metric challenging, even for a fixed
set of four points in two dimensions. In this paper, we overcome this
challenge by equating the Nearest Neighbor metric to a shortest-path
distance on a simple geometric graph, in all cases. Remarkably, this
equality holds even
if the point set is the countable union of compact geometric objects, which
are not necessarily convex or even simply connected. We
then compute a generalization of this metric, which we call the $q$-power
Nearest Neighbor metric, and prove an analagous equality for point sets
that are the union of 4 compact, path-connected geometric objects in
arbitrary dimension.\don{Something about Conformal change of metrics. If
you have a manifold and want to put a new Riemannian metric on it, usually
the Riemannian metric is a full-on metric tensor, but here we have a
simpler case where it's isotropic, same in every direction. In some sense,
these transformations are conformal, so this is a conformal change of
Riemannian metric. (This is what this general idea we're doing is on.
There's a bunch of literature on this, but almost nobody computes it.)}

Our proof uses conservative vector fields,
Lipschitz extensions, minimum cost flows, and
barycentric subdivisions, all applied to a geometric object we call the
$q$-screw simplex. This work considerably strengthens the work of Cohen et.
al., and shows the first non-trivial manifold metric that can be computed
exactly with discrete techniques. When the point set is finite, we can use
our results to solve a range of classical metric problems for our metric:
we can efficiently compute sparse spanners, compute persistent homology,
measure the behavior of the metric when the point set is a large number of
points drawn from an underlying probability density function, and show
links between this metrics and classic goemetric objects like Euclidean MST
or single-linkage clustering methods.

% The key geometric object in our proof, the $q$-screw simplex, was first
% discovered by John Von Neumann and Issai Schoenberg. This simplex is
% defined by taking $n$ points on a line, applying the $1/q$ power to all
% pairwise distances, and isometrically embedding the resulting point set
% into Euclidean space.  We prove results on these $q$-screw simplices that
% are of independent interest: we show that they
% have a deep connection to fractional Laplacians, a differential operator
% that appears in a wide variety of physics and mathematical settings, and
% isometrically embed these simplices into effective resistance distance.

% We use
%this connection to prove that the $q$-screw simplex embeds isometrically
%into Effective resistance distance, and that its volume and circumcenter
%can be approximated efficiently using a Laplacian determinant estimator and
%Laplacian system solver respectively.
%We also show that any finite $l_1$
%metric raised to the $1/q$ power is isometrically embeddable into $l_1$,
%mirroring a famous theorem by Schoenberg on $l_2$.
\end{abstract}
