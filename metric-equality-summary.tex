In this section, we prove our main theorems, Theorem~\ref{} and~\ref{}.
We prove that the edge-squared metric exactly equals the nearest neighbor
metric on any point set, and that the $q$-edge power metric equals the $q$-NN
metric for any set of four points or less. We also conjecture that the $q$-edge
power metric always equals the $q$-NN metric, and provide a discrete
inequality that would imply our larger conjecture. Even though the number of points
we handle for the general $q$-edge power metric is quite small, it still
solves a fairly difficult problem:
exactly computing the NN or $q$-NN metric even for four
points in two dimensions requires dealing with uncountably number of paths
through space.

Our tools for proving both theorems will be use of min-cost flows generated
from a conservative vector field. This is the core idea that lets us surmount
the difficulties in dealing with uncountably many paths. We hope that our ideas
may be more generally applicable to various metrics.  

Notice that the Nearest Neighbor cost is upper bounded by the edge-squared
metric. This can be done by purely considering Nearest Neighbor paths that
are piecewise linear and go straight from data point to data point.
Similarly, $q$-NN metrics are upper bounded by $q$-edge power metrics for
all $1$.
Therefore, we only need to prove that the Nearest Neighbor cost is lower
bounded by the edge-squared metric, and likewise for $q$-NN metrics.
To do this, we build a graph $G'$ from our point set, which can
conceptually be thought of as the edge-squared graph with additional
Steiner points. We will show using conservative vector fields and flows
that the Nearest Neighbor cost of any path from $a$ to $b$ is bounded below
by the shortest path from $a$ to $b$ in $G'$. If the shortest path in $G'$
were always equal to the shortest path in the edge-squraed graph $G$, then
we'd be done.

However, this is not the case in general. However, it does turn out to be
the case always on a $2$-screw simplex. Remarkably, we show that proving
this equality on the $2$ screw simplex is sufficient to prove it for any
point set, including point sets with uncountably many points. We show this
reduction via the Lipschitz extension theorem and a simple
BFS. This proves Theorem~\ref{thm:NN}

All our techniques generalize to $q$-NN metrics (when being related to $q$-edge power
metrics), except the equality
between shortest paths on $G'$ and $G$ is less clear for $q$-screw
simplices. We prove that this equality holds when there are four points in
three dimension, and conjecture (with computational evidence, but no proof) that this
equality holds for any point set. Doing this proves
Theorem~\ref{thm:qNN}, and provides a discrete criterion that would imply
Conjecture~\ref{conj:qNN} holds.

The gap in Theorem~\ref{thm:NN} and~\ref{thm:qNN} is due to the simple
geometric structure of the $2$-screw simplex, which is known to have a
