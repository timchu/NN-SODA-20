\section{Conclusions and Open Questions}\label{sec:conclusions}


We examined the nearest neighbor metric and showed how to compute it
exactly, as well as find sparse data structures efficiently for approximate
computatoin in practice.  Many problems remain open.

First: are there generalizations of these metrics, for which our proof techniques
will still hold? The nearest neighbor metric has many natural
generalizatoins, including the $k^{th}$ nearest neighbor or powers of the
nearest neighbor functoin.

Can we efficiently compute $o(\log n)$-spanners of the nearest neighbor
metric in high dimension, such the the spanners have a nearly linear number of edges?
The existence of such spanners has been studied for Euclidean metrics in~\cite{HarPeled13}, where the stretch
obtained is $\sqrt{\log n}$.

Does computing $k$-NN graphs with approximate nearest neighbor methods give $1$-spanners of the
edge-squared metric with high probability?
Approximate nearest neighbors have been studied extensively
~\cite{kNNsurvey, Chen11, Dong11}, including locality-sensitive
hashing for high dimensional point sets~\cite{LSH} and
more~\cite{Laarhoven2018}.
Recent work by Andoni et al.~\cite{Andoni2018} showed how to compute
approximate nearest neighbors for any non-Euclidean norm.  Perhaps there
is a rigorous theory about density-sensitive metrics generated from any such
norm? Similar to
how the edge-squared metric is generated from the Euclidean distance.

It remains an open question how well clustering or
classification with
nearest neighbor metric distances performs on real-world
data.
Experiments have been done by Bijral, Ratliff, and
Srebro in~\cite{bijral11semiSupLearningDBD}.
Theorem~\ref{thm:distribution-spanner} implies that future experiments
can be done using any
k-nearest-neighbor graph.


%%   We believe that the interest in alternative metrics on euclidean data will continue to be a rich source of interesting problems.
%%   We showed that the class of metrics that can be represented in this way is quite large, including all negative-type metrics and more.
%%   There is no constant upper bound on the doubling dimension of such metrics even for point sets in the plane.
%%   Theorem~\ref{thm:equality} extends these results immediately to density-based distances.
%%   Although there is such rich structure in these metrics, they still maintain some of the advantages of low-dimensional euclidean metrics, and this is exploited to generate sparse spanners.
%%   We also use the Euclidean structure to compute the persistent homology of the nearest neighbor distance in both the ambient sense and the intrinsic sense.
%%
%%   Many interesting problems remain open.
%%   One of the most compelling is the generalization of Theorem~\ref{thm:equality} to higher power metrics.
%%   It seems possible that integrating higher powers of the distance function could correspond exactly to higher power metrics.
%%   However, the proof techniques of this paper do not easily extend to that case.
%%   It does seem that as the power goes to infinity, shortest paths converge to the minimum spanning tree.
%%   For other powers in between $2$ and $\infty$, this question remains open.
