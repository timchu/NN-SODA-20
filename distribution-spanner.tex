\subsection{Exact-spanners of Nearest Neighbor Metric in
the Probability Density Setting
}\label{sec:distribution-spanner}
% \tim{Do a warm up on Gabriel graphs with convex support.}

Theorem~\ref{thm:distribution-spanner} states that for $k= O(2^d \log n)$, the $k$-NN graph of $n$ points drawn i.i.d from
a nicely behaved probability distribution is a $1$-spanner of the
Nearest Neighbor metric. This section is dedicatd to outlining a proof of
this Theorem, the full result which will be in
Appendix~\ref{ap:distribution-spanner}.
This result is clearly impossible for Euclidean distances,
whose $1$-spanner is the complete graph almost surely.  
Our theorem implies any off-the-shelf
$k$-nearest neighbor graph generator can compute
edge-squared metric. 

First, let us assume that the support of our
probability density $D$ has the same dimension as our ambient space.
This simplifies our calculations without changing the problem
much. Then, we note that as our number of sample points get
large, the density inside a $k$-NN ball around any point $x$ (the ball with radius
$k^{th}$-NN
distance, center at $x$) looks like the uniform distribution
on that ball, possibly intersected with a halfspace. The bounding
plane of our halfspace
represents the boundary of our density $D$.

For simplicity in the outline, let's suppose that $D$ is convex.
If we condition on
the radius of the $k$-NN ball, then the $k-1^{st}$ nearest
neighbors of $x$ are distributed roughly according to the above
distribution, described by the ball intersected with a halfspace.
For any other point $p$ in $D$, we project $p$ onto the $k$-NN
ball to point $p'$, and show that the ball $p'x$ contains a $k^{th}$ nearest
neighbor w.h.p, when $k=O(2^d \log n)$. This
implies ball with diameter $px$ contains a $k^{th}$ nearest
neighbor of $x$, and thus $px$ is not necessary in any
$1$-spanner of the edge-squared metric. Then we take union bound
over all $x$.
A rigorous proof of
Theorem~\ref{thm:distribution-spanner} requires careful analysis, 
and is contained in
Section~\ref{ap:distribution-spanner}.  
Our proof can be tweaked to show:
\begin{theorem}
  Given a Lipschitz distribution bounded above and below with support on convex set $C \subset
  \mathbb{R}^d$, the $k$-NN graph is Gabriel w.h.p. for $k =
  O(2^d \log n)$.
\end{theorem}
