\subsection{Reduction to $q$-screw simplex} % (fold)
\label{sec:reduction-to-screw}
In this section, we prove that it suffices to prove Theorem~\ref{thm:NN}
and~\ref{thm:qNN} on the $q$-screw simplex. We do so by doing it when
$q=2$, and noting that the proof generalizes naturally for any $q > 2$.

  Let $P\subset \R^d$ be a set of $n$ points.
  Pick any \emph{source} point $s\in P$.
  Order the points of $P$ as $p_1,\ldots ,p_n$ so that
  \[
    \dist_q(s,p_1) \le \cdots \le \dist_q(s, p_n).
  \]
  This will imply that $p_1 = s$.
  It will suffice to show that for all $p_i\in P$, we have $\dist_q(s,p_i)
= \dist_{qN}(s,p_i)$.
The core step is that we will find a Lipschitz map $m: \R \rightarrow \R^n$
that preserves $\dist_q(s,p)$ for all $p \in P$. We will then show how the
Lipschitz extension of $m$ is also Lipschitz as a function between $q$-NN
metrics. If we can do this, then the $q$-NN cost of any path $\gamma$ from $s$ to
$p_i$ on our initial point set is lower bounded by the $q$-NN cost of
$m(\gamma)$ on the point set $m(p_0), \ldots m(p_{n-1})$.
If the Theorem holds on $m(p_1), m(p_2), \ldots
m(p_{n-1})$, then this $q$-NN cost is lower bounded by the shortest path
from $m(p_0)$ to $m(p_{n-1})$, which is equal to the shortest path from
$p_0$ to $p_{n-1}$. This completes our reduction.
  
  \subsubsection{Generating a Lipschitz embedding into \R^n} % (fold)
\label{sec:lifting}
  
We seek to find points $m(p_i) \in \R^n$ such that 
\begin{align}\label{eq:preserve}
\dist_q(m(s,p_i),m(s,p_{i-1})) = \dist_q(s, p_i) - \dist_q(s,p_{i-1} )$
\end{align}

To find $m$, we perform a breadth-first search to find points on the real
line $x_0 < x_1 < \ldots x_{n-1}$ such that $x_i-x_{i-1} = \dist_q(s, p_i
- dist_q(s, p_{i-1})$. These points $x_i$ can be found with a simple
  breadth first search on our points. Note that if we set $m(p_0), \ldots
m(p_{n-1})$ as the vertices of the $q$-screw simplex
formed from points $x_0, x_1, \ldots x_n$, then Equation~\ref{eq:preserve}
holds.


% section lifting (end)

  \subsubsection{The Lipschitz Extension} % (fold)
\label{sec:lip_extension}

  Proposition~\ref{prop:m_is_good} and the Kirszbraun theorem on Lipschitz extensions imply that we can extend $m$ to a $1$-Lipschitz function $f: \R^d\to \R^n$ such that $f(p) = m(p)$ for all $p\in P$ \cite{Kirszbraun1934,Valentine1945,brehm1981}.

  \begin{lemma}\label{lem:dist_N_lipschitz}
    The function $f$ is also $1$-Lipschitz as mapping from $\R^d\to \R^n$ with both spaces endowed with the nearest neighbor geodesic.
  \end{lemma}
  \begin{proof}
    We are interested in two distance functions $\distto_P:\R^d \to \R$ and $\distto_{f(P)}: \R^n\to \R$.
    Recall that each is the distance to the nearest point in $P$ or $f(P)$ respectively.
    \begin{align*}
      \distto_{f(P)}(f(x)) 
        &= \min_{q\in f(P)} \|q - f(x)\| \because{by definition}\\
        &= \min_{p\in P} \|f(p) - f(x)\| \because{$q = f(p)$ for some $p$}\\
        &\le \min_{p\in P} \|p - x\| \because{$f$ is $1$-Lipschitz}\\ 
        &= \distto_P(x). \because{by definition}
    \end{align*}
    For any curve $\gamma:[0,1]\to \R^d$ and for all $t\in [0,1]$, we have $\|(f\circ \gamma)'(t)\| \le \|\gamma'(t)\|$.
    It then follows that
    \begin{equation}\label{eq:curves_shorten}
      \len'(f\circ \gamma) = \int_0^1 \distto_{f(P)}(f(\gamma(t)))\|(f\circ\gamma)'(t)\|dt \le \int_0^1 \distto_{P}(\gamma(t))\|\gamma'(t)\|dt = \len(\gamma),
    \end{equation}
    where $\len'$ denotes the length with respect to $\distto_{f(P)}$.
    Thus, for all $a,b\in P$,
    \begin{align*}
      \dist_N(a,b)
        &= 4 \inf_{\gamma\in \ourpath(a,b)} \len(\gamma) \because{by definition}\\
        &\ge 4 \inf_{\gamma\in \ourpath(a,b)} \len'(f\circ\gamma) \because{by \eqref{eq:curves_shorten}}\\
        &\ge 4 \inf_{\gamma'\in \ourpath(f(a),f(b))} \len'(\gamma') \because{because $f\circ\gamma\in \ourpath(f(a), f(b))$}\\
        &= \dist_N(f(a), f(b)). \because{by definition}
    \end{align*}
  \end{proof}
  
  \begin{theorem}\label{thm:equality}
    For any point set $P\subset\R^d$, the edge squared metric $\dist$ and the nearest neighbor geodesic $\dist_N$ are identical.
  \end{theorem}
  \begin{proof}
    Fix any pair of points $s$ and $p$ in $P$.
    Define the Lipschitz mapping $m$ and its extension $f$ as in \eqref{eq:defn_of_m}.
    Let $\dist'$ and $\dist_N'$ denote the edge-squared and nearest neighbor geodesics on $f(P)$ in $\R^n$.
    \begin{align*}
      \dist_2(s,p) 
        &= \dist'(m(s), m(p)) \because{Proposition~\ref{prop:m_is_good}(iii)}\\
        &= \dist_N'(m(s), m(p)) \because{$f(P)$ are vertices of a box}\\
        &\le \dist_N(s, p) \because{Lemma~\ref{lem:dist_N_lipschitz}}
    \end{align*}
    We have just shown that $\dist\le \dist_N$ and Lemma~\ref{lem:dist_N_le_dist} states that $\dist\ge \dist_N$, so we conclude that $\dist = \dist_N$ as desired.
  \end{proof}

% section lip_extension (end)

  
% section the_proof (end)
