\subsection{Reduction to $q$-screw simplex} % (fold)
\label{sec:reduction-to-screw}
In this section, we prove that it suffices to prove Theorem~\ref{thm:NN}
and~\ref{thm:qNN} on the $q$-screw simplex. We do so by doing it when
$q=2$, and noting that the proof generalizes naturally for any $q > 2$.

  Let $P\subset \R^d$ be a set of $n$ points.
  Pick any \emph{source} point $s\in P$.
  Order the points of $P$ as $p_1,\ldots ,p_n$ so that
  \[
    \dist_q(s,p_1) \le \cdots \le \dist_q(s, p_n).
  \]
  This will imply that $p_1 = s$.
  It will suffice to show that for all $p_i\in P$, we have $\dist_2(s,p_i) = \dist_N(s,p_i)$.
  There are three main steps:
  \begin{enumerate}
  \item We first show that the theorem holds for $2$-screw simplices, and
for $q$-screw simplices when $q > 2$.
\begin{enumerate}
  \item We will construct a graph $G'$ on the points $x_0, x_1, \ldots x_n$
of the $q$-screw simplex, and prove that the $q$-NN distance between $p_1$
and $p_n$ is bounded above by the shortest path distance between
corresponding vertices on $G'$.
	\item We show that the shortest path on $G'$ and the shortest path
on $G$ coincide when $q=2$, and for $4$ point sets when $q>2$.  
\end{enumerate}
  \item We then show how to lift the points from $\R^d$ to $\R^n$ by a
Lipschitz map $m$ that places all the points on the vertices of a $q$-screw
simplex, and preserves $\dist_q(s,p)$ for all $p\in
P$\tim{define $\dist_q$}
  \item We show how the Lipschitz extension of $m$ is also
Lipschitz as a function between $q$-power Nearest Neighbor metrics. 
We combine these pieces to show that $\dist_2 \le \dist_N$.  As $\dist_2
\ge \dist_N$ (Lemma~\ref{lem:dist_N_le_dist}), this will conclude the proof
that $\dist_2 = \dist_N$. The analagous result holds for $4$ point sets in
the $q$-NN metric, which concludes the proof that $\dist_q = \dist_{qN}$
for all $q>2$ for $4$ point sets.
  \end{enumerate}
  
  \input{boxes}
  \subsubsection{Generating a Lipschitz embedding into \R^n} % (fold)
\label{sec:lifting}
  
We seek to find points $m(p_i) \in \R^n$ such that 
\begin{align}\label{eq:preserve}
\dist_q(m(s,p_i),m(s,p_{i-1})) = \dist_q(s, p_i) - \dist_q(s,p_{i-1} )$
\end{align}

To find $m$, we perform a breadth-first search to find points on the real
line $x_0 < x_1 < \ldots x_{n-1}$ such that $x_i-x_{i-1} = \dist_q(s, p_i
- dist_q(s, p_{i-1})$. These points $x_i$ can be found with a simple
  breadth first search on our points. Note that if we set $m(p_0), \ldots
m(p_{n-1})$ as the vertices of the $q$-screw simplex
formed from points $x_0, x_1, \ldots x_n$, then Equation~\ref{eq:preserve}
holds.


% section lifting (end)

  \subsubsection{The Lipschitz Extension} % (fold)
\label{sec:lip_extension}

  Proposition~\ref{prop:m_is_good} and the Kirszbraun theorem on Lipschitz
extensions imply that we can extend $m$ to a $1$-Lipschitz function $f:
\R^d\to \R^n$ such that $f(p) = m(p)$ for all $p\in P$
\cite{Kirszbraun1934,Valentine1945,brehm1981}.

  \begin{lemma}\label{lem:dist_N_lipschitz}
    The function $f$ is also $1$-Lipschitz as mapping from $\R^d\to \R^n$
with both spaces endowed with the $q$-NN metric.
  \end{lemma}
  \begin{proof}
    We are interested in two distance functions $\distto_P:\R^d \to \R$ and $\distto_{f(P)}: \R^n\to \R$.
    Recall that each is the distance to the nearest point in $P$ or $f(P)$ respectively.
    \begin{align*}
      \distto_{f(P)}(f(x)) 
        &= \min_{q\in f(P)} \|q - f(x)\| \because{by definition}\\
        &= \min_{p\in P} \|f(p) - f(x)\| \because{$q = f(p)$ for some $p$}\\
        &\le \min_{p\in P} \|p - x\| \because{$f$ is $1$-Lipschitz}\\ 
        &= \distto_P(x). \because{by definition}
    \end{align*}
    For any curve $\gamma:[0,1]\to \R^d$ and for all $t\in [0,1]$, we have $\|(f\circ \gamma)'(t)\| \le \|\gamma'(t)\|$.
    It then follows that
    \begin{equation}\label{eq:curves_shorten}
      \len'(f\circ \gamma) = \int_0^1 \distto_{f(P)}(f(\gamma(t)))\|(f\circ\gamma)'(t)\|dt \le \int_0^1 \distto_{P}(\gamma(t))\|\gamma'(t)\|dt = \len(\gamma),
    \end{equation}
    where $\len'$ denotes the length with respect to $\distto_{f(P)}$.
    Thus, for all $a,b\in P$,
    \begin{align*}
      \dist_{qN}(a,b)
        &= q \inf_{\gamma\in \ourpath(a,b)} \len(\gamma) \because{by definition}\\
        &\ge q \inf_{\gamma\in \ourpath(a,b)} \len'(f\circ\gamma) \because{by \eqref{eq:curves_shorten}}\\
        &\ge q \inf_{\gamma'\in \ourpath(f(a),f(b))} \len'(\gamma') \because{because $f\circ\gamma\in \ourpath(f(a), f(b))$}\\
        &= \dist_{qN}(f(a), f(b)). \because{by definition}
    \end{align*}
  \end{proof}

% section lip_extension (end)

  
% section the_proof (end)
